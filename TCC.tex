\documentclass[a4paper]{article}

\usepackage[brazil]{babel}
\usepackage[utf8]{inputenc}
\usepackage{amsmath,amsfonts,amssymb,latexsym,mathrsfs,amsthm,amstext,bezier,amscd}
\usepackage{graphicx}
\usepackage{indentfirst}
\usepackage{setspace}
\usepackage{longtable}
\usepackage{tikz}
\usepackage{geometry}
\usepackage{chngcntr}
\usetikzlibrary{shapes,backgrounds}

\begin{document}
\thispagestyle{empty}

\begin{titlepage}
	\vfill
	\begin{center}
		\onehalfspacing
		\parbox{6cm}{\includegraphics[scale=0.2]{logo.png}}\\
		\begingroup
		\fontsize{12pt}{0pt}\selectfont
		{\large \textbf{INSTITUTO FEDERAL DE EDUCAÇÃO, CIÊNCIA E TECNOLOGIA DO CEARÁ}}\\[0.2cm]
		\fontsize{12pt}{0pt}\selectfont
		{\large \textbf{IFCE CAMPUS MARACANAÚ}}\\[0.2cm]
		\fontsize{12pt}{0pt}\selectfont	
		{\large \textbf{BACHARELADO EM CIÊNCIA DA COMPUTAÇÃO}}\\[3.5cm]
		\fontsize{12pt}{0pt}\selectfont
		{\large \textbf{THIAGO MAGALHÃES FURTADO}}\\[3.5cm]
		\fontsize{12pt}{0pt}\selectfont
		{\large \textbf{AVALIAÇÃO DA ACESSIBILIDADE EM PORTAIS DE NOTÍCIAS}}\\[3.5cm]
		\fontsize{12pt}{0pt}\selectfont
		{\large \textbf{MARACANAÚ}}\\[0.2cm]
		\fontsize{12pt}{0pt}\selectfont
		{\large \textbf{2021}}
		\endgroup
	\end{center}
\end{titlepage}

\begin{titlepage}
	\vfill
	\begin{center}
		\onehalfspacing
		\fontsize{12pt}{0pt}\selectfont
		{\large \textbf{THIAGO MAGALHÃES FURTADO}} \\[1.5cm]
		\fontsize{12pt}{0pt}\selectfont
		{\large \textbf{AVALIAÇÃO DA ACESSIBILIDADE EM PORTAIS DE NOTÍCIAS}}\\[3cm]
		
		\hspace{.45\textwidth} %posiciona a minipage
		\begin{minipage}{.5\textwidth}
			\large Trabalho de Conclusão de Curso apresentado ao curso de Bacharelado em Ciência da Computação do Instituto Federal de Educação, Ciência e Tecnologia do Ceará (IFCE) - Campus Maracanaú, como requisito parcial para obtenção do Título de Bacharel em Ciência da Computação.\\[1cm]
			Prof. Dr. Otávio Alcântara de Lima Junior.
		\end{minipage}
		\vfill
		\vspace{2cm}		
		\large \textbf{Maracanaú - CE}
		
		\large \textbf{2021}
	\end{center}
\end{titlepage}

\begin{titlepage}
	\vfill
	\begin{center}
		\onehalfspacing
		\fontsize{12pt}{0pt}\selectfont
		{\large \textbf{THIAGO MAGALHÃES FURTADO}} \\[1.5cm]
		\fontsize{12pt}{0pt}\selectfont
		{\large \textbf{AVALIAÇÃO DA ACESSIBILIDADE EM PORTAIS DE NOTÍCIAS}}\\[3cm]
		
		\hspace{.45\textwidth} %posiciona a minipage
		\begin{minipage}{.5\textwidth}
			\large Trabalho de Conclusão de Curso apresentado ao curso de Bacharelado em Ciência da Computação do Instituto Federal de Educação, Ciência e Tecnologia do Ceará (IFCE) - Campus Maracanaú, como requisito parcial para obtenção do Título de Bacharel em Ciência da Computação.\\[1cm]
		\end{minipage}
	\end{center}
	
	Aprovado em:  \rule{1cm}{0.01cm} /\rule{1cm}{0.01cm} /\rule{1cm}{0.01cm}.\\[0.5cm]
	
	\begin{center}
		\onehalfspacing
		\fontsize{12pt}{0pt}\selectfont
		{\large{BANCA EXAMINADORA}}\\[1cm]
		\rule{13cm}{0.01cm}\\[0.1cm]
		Prof. Dr. Otávio Alcântara de Lima Junior (Orientador)\\[0.1cm]
		Instituto Federal de Educação, Ciência e Tecnologia do Ceará (IFCE)\\[1.5cm]
		\rule{13cm}{0.01cm}\\[0.1cm]
		Prof. \\[0.1cm]
		Instituto Federal de Educação, Ciência e Tecnologia do Ceará (IFCE)\\[1.5cm]
		\rule{13cm}{0.01cm}\\[0.1cm]
		Prof. \\[0.1cm]
		Instituto Federal de Educação, Ciência e Tecnologia do Ceará (IFCE)\\
	\end{center}
\end{titlepage}

\newgeometry{top=22cm}
\begin{titlepage}
	
	\hfill \parbox{7.5cm}{A Deus.}
	
	\hfill \parbox{7.5cm}{Ao Senhor Jesus Cristo.}
	
	\hfill \parbox{7.5cm}{A minha igreja.}
	
	\hfill \parbox{7.5cm}{Aos meus pais.}

	\hfill \parbox{7.5cm}{Aos meus amigos.}
	
	\hfill \parbox{7.5cm}{Aos meus professores e mestres.}
\end{titlepage}

\restoregeometry
\begin{titlepage}
	\begin{center}
		{\large \textbf{AGRADECIMENTOS}}\\[1cm]
	\end{center}

	A Deus, por tudo.
	
	À minha família, pelo incentivo.
	
	À minha igreja, pelas orações, pelo companheirismo e por estarem ao meu lado em toda a minha jornada da graduação.
	
	Aos amigos e colegas de estudo, que vivenciaram comigo os desafios e me ajudaram a vencê-los. Agradeço o carinho, o apoio, o acolhimento, a paciência, os conselhos, os ensinamentos e as palavras motivadoras.
	
	Aos professores, que muito contribuíram com minha formação acadêmica, agradeço os ensinamentos, as orientações, as lições de vida, os risos, a atenção. Vocês são verdadeiros mestres.	
\end{titlepage}

\begin{titlepage}
	\begin{center}
		{\large \textbf{RESUMO}}\\[1cm]
	\end{center}
	\fontsize{12pt}{0pt}\selectfont
	\onehalfspacing
	Atualmente, existe uma necessidade muito grande da conscientização da acessibilidade para deficientes físicos, inclusive com a acessibilidade digital. Há muitas pessoas, no mundo, que possuem deficiência ou que tem algum obstáculo que impossibilita o acesso e o manuseio dos serviços oferecidos pela web. Dentre esses obstáculos, existem pessoas com baixa visão, com daltonismo, com pouca audição, com surdez parcial ou total, com diminuição da capacidade motora ou, ainda, pessoas com deficiência mental, dentre outras. Um dos ramos de sites são as plataformas de notícia, que são de grande importância nos dias atuais, pois eles são um dos meios de informação existentes para a população do Brasil e do Mundo, tornado-se necessário uma avaliação da acessibilidade digital deles. Assim, essa monografia propõe uma avaliação de acessibilidade de dez dos principais sites de notícias brasileiros. Primeiramente, foram descritas duas diretrizes usadas na atualidade, uma é a eMAG, isto é, Modelo de Acessibilidade em Governo Eletrônico, e a outra é a WCAG, ou seja, Web Content Accessibility Guidelines, em português, visão geral das diretrizes de acessibilidade de conteúdo da web. Portanto, essa monografia tem como objetivo analisar e verificar a acessibilidade das plataformas, com a finalidade de quantificá-los para concluir o grau de acessibilidade, individualmente, de dez sites de notícias, usando o estudo feito das diretrizes da WCAG. A metodologia proposta é estruturada em seis etapas. Primeiro, foram selecionadas as diretrizes da WCAG que tem relação com a deficiência visual e com a deficiência auditiva, descrevendo cada uma delas, individualmente. Após isso, foi especificado o método de computo da acessibilidade a partir do atendimento as diretrizes, que será usado nos sites. A seguir, foram escolhidos dez sites através de um determinado critério. Posteriormente foi feita uma avaliação manual da acessibilidade de cada site de acordo com as especificações, usando o método escolhido. Outrossim, foi coletado os dados da análise do levantamento feito proveniente da análise dos sítios. Por fim, foi feito uma análise comparativa, com a finalidade de chegar a um determinado resultado final. A partir dos resultados, pode-se afirmar que essas plataformas digitais não são acessíveis para pessoas com deficiência visual e nem para as pessoas com deficiência auditiva, pois, de acordo com a averiguação feita, apenas (Porcentagem à colocar)\% das diretrizes nos dez sites foram cumpridas, concluindo que se faz necessário um grande esforço de todos os profissionais envolvidos na criação dos veículos de notícia em torno do seu conteúdo acessível para pessoas com deficiência.\\[1cm]
	{\large \textbf{Palavras-chave: }}Acessibilidade. Pessoas com deficiência. WCAG. Sites de notícias.\\[1cm]
\end{titlepage}

\begin{titlepage}
	\begin{center}
		{\large \textbf{ABSTRACT}}\\[1cm]
	\end{center}
	\fontsize{12pt}{0pt}\selectfont
	\onehalfspacing
	Currently, there is a great need for awareness of accessibility for the physically disabled, including digital accessibility. There are many people in the world who have a disability or who have an obstacle that makes it impossible to access and handle the services offered on the web. Among these obstacles, there are people with low vision, color blindness, with little hearing, with partial or total deafness, with reduced motor capacity, or even people with mental disabilities, among others. One of the branches of websites are news platforms, which are of great importance nowadays, as they are one of the existing means of information for the population of Brazil and the world, making an assessment of their digital accessibility necessary. Thus, this monograph proposes an accessibility assessment of ten of the main Brazilian news sites. First, two guidelines used today were described, one is the eMAG, that is, the Accessibility Model in Electronic Government, and the other is the WCAG, that is, the Web Content Accessibility Guidelines, in Portuguese, an overview of the accessibility guidelines of Web content. Therefore, this monograph aims to analyze and verify the accessibility of platforms, in order to quantify them to conclude the degree of accessibility, individually, of ten news sites, using the study made of the WCAG guidelines. The proposed methodology is structured in six stages. First, WCAG guidelines related to visual impairment and hearing impairment were selected, describing each one individually. After that, the method of computing accessibility was specified based on the compliance with the guidelines, which will be used on the websites. Next, ten sites were chosen according to certain criteria. Subsequently, a manual assessment of the accessibility of each site was carried out according to the specifications, using the chosen method. Furthermore, data from the analysis of the survey carried out from the analysis of the sites were collected. Finally, a comparative analysis was carried out, in order to reach a certain final result. From the results, it can be stated that these digital platforms are not accessible for people with visual impairments or for people with hearing impairments, as, according to the investigation carried out, only (Percentage to be placed)\% of the guidelines in the ten sites were completed, concluding that a great effort is needed by all professionals involved in the creation of news outlets around their content accessible to people with disabilities.\\[1cm]
	{\large \textbf{Keywords: }}Accessibility. Disabled people. WCAG. News sites.\\[1cm]
\end{titlepage}

\begin{titlepage}
	\begin{center}
		{\large \textbf{LISTA DE ILUSTRAÇÕES}}\\[1cm]
	\end{center}
	Quadro 1 — Diretrizes do princípio perceptível.......................................................................\hspace{0.4cm}4\\[0.5cm]
	Quadro 2 — Diretrizes do princípio operável...........................................................................\hspace{0.4cm}5\\[0.5cm]
	Quadro 3 — Diretrizes do princípio compreensível..................................................................\hspace{0.4cm}6\\[0.5cm]
	Quadro 4 — Diretrizes do princípio robusto............................................................................\hspace{0.4cm}7\\[0.5cm]
	Figura 1 — Fluxograma das etapas da metodologia..................................................................\hspace{0.2cm}16\\[0.5cm]
	Quadro 5 — Especificações da WCAG relacionadas com a deficiência visual...........................\hspace{0.2cm}17\\[0.5cm]
	Quadro 6 — Especificações da WCAG relacionadas com a deficiência auditiva.......................\hspace{0.2cm}21\\[0.5cm]
\end{titlepage}

\begin{titlepage}
	\begin{center}
		{\large \textbf{LISTA DE TABELAS}}\\[1cm]
	\end{center}
	Tabela 1 — Diretrizes da WCAG analisadas nos sites..............................................................\hspace{0.2cm}10\\[0.5cm]
	Tabela 2 — Deficiências visuais e as diretrizes da WCAG........................................................\hspace{0.2cm}13\\[0.5cm]
	Tabela 3 — Deficiências auditiva e as diretrizes da WCAG......................................................\hspace{0.2cm}14\\[0.5cm]
	Tabela 4 — Diretrizes escolhidas e seus pesos..........................................................................\hspace{0.2cm}23\\[0.5cm]
	Tabela 5 — Diretrizes cumpridas nos sites...............................................................................\hspace{0.2cm}28\\[0.5cm]
\end{titlepage}

\begin{titlepage}
	\begin{center}
		\tableofcontents
	\end{center}
\end{titlepage}

\begin{titlepage}
\section{INTRODUÇÃO}
\fontsize{12pt}{0pt}\selectfont
\onehalfspacing

\subsection{Introdução}

Estamos vivendo em uma época em que a tecnologia tem avançado bastante, isso é notório se nós analisarmos as três últimas décadas, durante as quais o mundo entrou na chamada era da informação. Assim, foram feitos avanços significativos, como o aumento do acesso à Internet e à Tecnologia da informação por uma parcela significativa da população mundial. Esse aumento possibilita, para a nossa sociedade, uma prestação de serviço e uma obtenção de informação rápida, tornando-se importantíssimo. Porém, muitos sites não possibilitam uma navegação agradável e tranquila a todos, já que pessoas com deficiência precisam que o conteúdo dos sites estejam adaptados à suas necessidades. Segundo a OMS, 15\% da população mundial tem alguma deficiência. Essas pessoas possivelmente enfrentam dificuldades para navegar na Internet. [1].

Segundo a Organização das Nações Unidas, em 2011, impedir o acesso à Internet por meio de tecnologias sofisticadas afronta o  parágrafo 3º do artigo 19 do Pacto Internacional sobre os Direitos Civis e Políticos, violando o direito à liberdade de expressão. Este Pacto foi promulgado no Brasil em 1992 pelo decreto de número 592. Assim, toda pessoa tem direito à liberdade de expressão, incluindo a liberdade de procurar, receber e difundir notícias ou pensamentos de qualquer natureza, inclusive, informação e ideias disponibilizadas na Internet [35].

\subsection{Motivação}

Dentre a diversidade de conteúdo da Internet, pode-se destacar os portais de notícias como ferramentas importantes para divulgação de informação de qualidade para a sociedade. Porém essas plataformas digitais precisam alcançar todos os usuários, inclusive os PCDs (pessoas com deficiência). Para avaliar a acessibilidade desses sites, é necessário a aplicação de diretrizes de acessibilidade que são propostas por instituições e governos. Por exemplo, a eMAG é uma diretriz proposta pelo governo brasileiro em 2004, utilizada por algumas plataformas digitais nacionais [2].

Outra diretriz usada atualmente é a WCAG, Web Content Accessibility Guidelines, que já está na sua segunda versão, isto é, a WCAG 2.0. A WCAG possuí doze diretrizes organizadas em quatro princípios diferentes, possuindo algumas recomendações para cada princípio [3], que são indispensáveis em qualquer plataforma digital, inclusive nos sites de notícia.

O WCAG é um padrão adotado em diversos trabalhos de avaliação da acessibilidade. Em [4], foi avaliado a acessibilidade dos vídeos na plataforma online de curso MOOC. Foram analisados os vídeos através da verificação do uso de uma alternativa à informação visual, de audiodescrição, transcrições, de uma fonte boa, com um tamanho de letra razoável e de alto-contraste. Em [5] e em [6] foi analisado a existência de acessibilidade dos sites de ensino do nível superior da Índia, utilizando o WCAG e uma ferramenta de avaliação, chamada de TAW. Em 2016, foi feito um acompanhamento, por meio da WCAG, dos serviços de governo eletrônico da Arábia Saudita, com o objetivo de analisar o nível de conscientização e as políticas do próprio governo [7]. Por fim, foi feito uma análise de um método heurístico existente para investigar o nível de acessibilidade de 40 sites em relação aos usuários com baixa visão juntamente com a diretriz WCAG 2.1, o método utilizado foi proposto por Brajnik [8].

\subsection{Objetivos}

Esta monografia propõe a avaliação da acessibilidade de dez dos principais portais de notícias brasileiros, que têm um papel importantíssimo em relação ao repasse de informação para o público em geral, inclusive para os usuários PCDs, supondo que eles não atendem de forma plena os requisitos desse público. Para isso é proposto uma avaliação da acessibilidade dos sites para as pessoas que possuem algum grau de deficiência visual ou auditiva. 

\subsubsection{Objetivo geral}

A pesquisa tem o intuito de analisar os sites para quantificar o grau de acessibilidade de cada um. Assim, o objetivo geral do trabalho é avaliar sites de notícias que possuem formas de acessibilidade digital, sejam completamente, parcialmente ou não acessíveis para a deficiência visual e para a deficiência auditiva. 

\subsubsection{Objetivos específicos}

A metodologia empregada consiste em caracterizar a definição de acessibilidade digital, verificar as formas de deficiência existentes que afetam a usabilidade dos sites, apresentando a diretriz WCAG, escolhendo os princípios destinados aos deficientes visuais e auditivos para mensurar os sites de notícias em relação a acessibilidade.

\subsection{Estrutura}

O restante da monografia está organizado da seguinte forma.
Na capítulo 2 é apresentado duas diretrizes usadas na atualidade, que são a eMAG e a WCAG. Além disso, será relatado o estado da arte de avaliação da acessibilidade em sites usando as diretrizes, mostrando quais diretrizes usaram e em quais plataformas aconteceram as suas pesquisas. Por último, nesse mesmo capítulo, será explicado as formas de deficiência visual e auditiva.
Na capítulo 3 é mostrado a metodologia que será aplicada para fazer a averiguação dos sites.
Na capítulo 4 é feita a análise dos sites de notícias relacionado com a acessibilidade web, enfatizando os pontos fortes e fracos, através da análise das diretrizes escolhidas da WCAG, com o propósito de expor o nível da acessibilidade das plataformas digitais.
Por fim, na capítulo 5 é apresentada uma conclusão desse trabalho.

\section{REFERENCIAL TEÓRICO}
Existe uma grande produção de trabalhos relacionados com acessibilidade web, contudo serão concentrados os artigos publicados, principalmente, em revistas acadêmicas, como por exemplo, IEEEAcess, Universal Access in the Information Society, Springer, Electronic Government an International
Journal, entre outras, concentrados no período de 2013 até o ano de 2021. O restante do capítulo está organizado da seguinte maneira, são mostrados os artigos que tratam sobre as diretrizes utilizadas nos respectivos estudos e sobre as propostas de avaliação de acessibilidade em diversas áreas e, após isso, será detalhado, de forma sucinta, as mais variadas formas de deficiência visual e auditiva.

\subsection{Diretrizes}
Sobre as diretrizes, serão apresentados alguns trabalhos que tratam sobre a diretriz WCAG e a diretriz eMAG.

\subsubsection{WCAG}

A WCAG, publicada pela World Wide Web (W3C) através da organização chamada World Wide Consortium, que é um consórcio com 450 membros, empresas, organizações independentes e ONGs, tem o objetivo de formular padrões para que os desenvolvedores criar sites acessíveis. A WCAG 2.0 tem a finalidade de eliminar erros de acessibilidade através dos \textit{web designers} e dos desenvolvedores, por meio de recomendações e de princípios [3], que estão detalhados na sua documentação [10]. Assim, ela serve de apoio para os desenvolvedores a boa prática da acessibilidade web, garantindo um acesso à internet satisfatório, atingindo o maior número de pessoas possível, independente da existência de uma comorbidade física.

\paragraph{1 - Princípio perceptível:}

O primeiro princípio da WCAG é o perceptível, no qual a informação e os componentes da interface de utilizador precisam ser apresentados de forma para que os usuários possam entender. Nesse princípio existem quatro diretrizes.

Na primeira diretriz diz que o site tem que ter alternativas em texto, isto é, ele deve fornecer as alternativas para todo o conteúdo não textual de modo que possa ser apresentado de outras formas, segundo as necessidades dos utilizadores. Alguns exemplos são caracteres ampliados, braille, fala, símbolos ou uma linguagem mais simples. A segunda é o site ter uma mídia dinâmica ou contínua, remetendo que ele deve fornecer alternativas para conteúdo em multimídia dinâmica ou temporal. A terceira diretriz é o adaptável, que diz que deve ser criado um conteúdo que possa ser apresentado de diferentes formas, por exemplo, um esquema de página mais simples, sem perder a informação ou a estrutura. Por fim, a diretriz distinguível expressa que deve haver uma facilitação dos utilizadores à audição e à visão dos conteúdos nomeadamente através da separação do primeiro plano do plano de fundo. Todas as diretrizes do princípio perceptível estão especificadas e descritas no quadro 1.\\

Quadro 1 - Diretrizes do princípio perceptível\\[-1cm]
\begin{center}
	\begin{longtable}{|c|c|}
		\hline
		Diretrizes & Especificações \\
		\hline
		Diretriz 1.1 Alternativas& 1.1.1 Conteúdo Não Textual\\
		em Texto & \\
		\hline
		Diretriz 1.2 Mídia & 1.2.1 Conteúdo só de áudio\\
		Dinâmica ou Contínua & e só de vídeo (pré-gravado) \\
		& 1.2.2 Legendas (pré-gravadas)\\
		& 1.2.3 Audiodescrição ou Alternativa\\
		& em Multimídia (pré-gravada)\\
		& 1.2.4 Legendas (em direto)\\
		& 1.2.5 Audiodescrição (pré-gravada)\\
		& 1.2.6 Língua Gestual (pré-gravada)\\
		& 1.2.7 Audiodescrição Alargada (pré-gravada)\\
		& 1.2.8 Alternativa em Multimídia (pré-gravada)\\
		& 1.2.9 Só áudio (em direto)\\
		\hline
		Diretriz 1.3 Adaptável& 1.3.1 Informações e Relações\\
		& 1.3.2 Sequência com Significado\\
		& 1.3.3 Características Sensoriais\\
		\hline
		Diretriz 1.4 Distinguível& 1.4.1 Utilização da Cor\\
		& 1.4.2 Controle de Áudio\\
		& 1.4.3 Contraste\\
		& 1.4.4 Redimensionar texto\\
		& 1.4.5 Imagens de texto\\
		& 1.4.6 Contraste (Melhorado)\\
		& 1.4.7 Som Baixo ou Ausência de Som de Fundo\\
		& 1.4.7 Ausência de Fundo, Desligar\\
		& 1.4.8 Apresentação Visual\\
		& 1.4.9 Imagens de texto (sem exceção)\\
		\hline
	\end{longtable}
\end{center}

\paragraph{2 - Princípio operável: }

O segundo princípio é o operável, no qual é definido os métodos, que possibilitam, para os usuários, uma interação e uma boa navegação pelo conteúdo de forma confortável. O princípio diz que os sites devem usar interfaces apropriadas para os deficientes, isto é, os componentes da interface de utilizador e a navegação têm de ser operáveis. Nesse princípio existem quatro diretrizes com suas especificações.

As diretrizes determinam, primeiramente, que o site deve ser acessível por teclado, isto é, ele deve ter uma funcionalidade que fique disponível a partir do teclado. Além disso, ele deve ter um tempo suficiente, revelando que ele deve proporcionar aos utilizadores um tempo suficiente para lerem e para utilizarem o conteúdo. A terceira determinação é que o site não deve causar convulsões, pressupondo-se que ele não deve criar conteúdos de uma forma que se sabe que pode causar convulsões em alguns usuários. E, por fim, o site deve ser navegável, representando que ele deve fornecer formas de ajudar os utilizadores a navegar, localizar conteúdos e determinar o local onde estão. Todas as diretrizes do princípio operável estão especificadas e descritas no quadro 2.\\

Quadro 2 - Diretrizes do princípio operável\\[-1cm]
\begin{center}
	\begin{longtable}{|c|c|}
		\hline
		Diretrizes & Especificações \\
		\hline
		Diretriz 2.1 Acessível por Teclado& 2.1.1 Teclado\\
		& 2.1.2 Sem bloqueio do teclado\\
		& 2.1.3 Teclado(Sem Exceção)\\
		\hline
		Diretriz 2.2 Tempo Suficiente & 2.2.1 Tempo ajustável \\
		& 2.2.2 Colocar em pausa, parar, ocultar\\
		& 2.2.3 Sem temporização\\
		& 2.2.4 Interrupções\\
		& 2.2.5 Nova autenticação\\
		\hline
		Diretriz 2.3 Convulsões& 2.3.1 Três Flashes ou Abaixo do Limite\\
		& 2.3.2 Três Flashes\\
		\hline
		Diretriz 2.4 Navegável& 2.4.1 Ignorar Blocos\\
		& 2.4.2 Página com Título\\
		& 2.4.3 Ordem do Foco\\
		& 2.4.4 Finalidade da\\
		& Hiperligação (Em Contexto)\\
		& 2.4.5 Várias Formas\\
		& 2.4.6 Cabeçalhos e Etiquetas\\
		& 2.4.7 Foco Visível\\
		& 2.4.8 Localização\\
		& 2.4.9 Finalidade da Hiperligação\\
		& 2.4.9 (Apenas a Hiperligação)\\
		& 2.4.10 Cabeçalhos da Secção\\
		\hline
	\end{longtable}
\end{center}

\paragraph{3 - Princípio compreensível: }

O terceiro princípio é o compreensível, que diz que os usuários devem ser capazes de compreender as informações e o funcionamento da interface, com o objetivo do usuário interpretar corretamente o conteúdo da plataforma digital, isto é, a informação e a utilização da interface de utilizador têm de ser compreensíveis. Nesse princípio existem três diretrizes com suas especificações.

As diretrizes determinam que o site deve ser legível, simbolizando que ele é capaz de tornar o conteúdo textual legível e compreensível. Ele deve ser previsível, refletindo a ideia de fazer com que as páginas Web apareçam e funcionem de forma previsível, devendo existir assistência na inserção de dados, retratando que ela deve ajudar os utilizadores a evitar e a corrigir os erros. Todas as diretrizes do princípio compreensível estão especificadas e descritas no quadro 3.\\

Quadro 3 - Diretrizes do princípio compreensível\\[-1cm]
\begin{center}
	\begin{longtable}{|c|c|}
		\hline
		Diretrizes & Especificações \\
		\hline
		Diretriz 3.1 Legível& 3.1.1 Idioma da página\\
		& 3.1.2 Idioma das partes\\
		& 3.1.3 Palavras invulgares\\
		\hline
		Diretriz 3.2 Previsível & 3.2.1 Ao receber o Foco\\
		& 3.2.2 Ao entrar num campo\\
		& de edição (input)\\
		& 3.2.3 Consistência de Navegação\\
		& 3.2.4 Consistência de Identificação\\
		& 3.2.5 Alteração a Pedido\\
		\hline
		Diretriz 3.3 Assistência na Inserção de Dados& 3.3.1 Identificação de Erros\\
		& 3.3.2 Etiquetas ou Instruções\\
		& 3.3.3 Sugestão para eliminar o Erro\\
		& 3.3.4 Prevenção de Erros\\
		& (Legais, Financeiros, Dados)\\
		& 3.3.5 Ajuda\\
		& 3.3.6 Prevenção de Erros\\
		& (de qualquer tipo)\\
		\hline
	\end{longtable}
\end{center}

\paragraph{4 - Princípio robusto: }

O quarto princípio é o robusto, que foca na grande variedade dos usuários poderem interpretar o conteúdo da maneira mais confiável possível, além de levar, em consideração, a compatibilidade com as tecnologias atuais e futuras. Isso tem a finalidade de maximizar a harmonização das páginas da web com os seus usuários e com essas tais tecnologias, isto é, o conteúdo deve ser suficientemente robusto para ser interpretado de forma fiável por uma ampla variedade de agentes utilizadores do site em questão, incluindo as tecnologias de apoio. Nesse princípio existe apenas uma diretriz com suas especificações.

A diretriz determina que a plataforma digital deve ser compatível, isto é, ele deve maximizar a compatibilidade com os agentes utilizadores atuais e futuros, incluindo as tecnologias de apoio. Essa única diretriz presente no princípio compreensível está especificada e descrita no quadro logo abaixo.\\

Quadro 4 - Diretrizes do princípio robusto\\[-1cm]
\begin{center}
	\begin{longtable}{|c|c|}
		\hline
		Diretrizes & Especificações \\
		\hline
		Diretriz 4.1 Compatível& 4.1.1 Análise sintática (parsing)\\
		& 4.1.2 Nome, Função, Valor\\
		\hline
	\end{longtable}
\end{center}

\subsubsection{eMAG}

Em 2004, no Brasil, foi publicado e disponibilizado pelo governo, o Modelo de Acessibilidade em Governo Eletrônico (eMAG). Esse é o principal documento de recomendações a serem seguidas para o desenvolvimento de sítios eletrônicos acessíveis a todos os usuários do país. O objetivo da padronização dos portais públicos federais é minimizar os problemas de geração de conteúdo e de utilização das páginas Web, tornando-os mais acessíveis e facilitando o acesso à informação para todos os usuários, independente de possuírem ou não alguma deficiência [2].

Existe uma documentação da eMAG que mostra as padronizações e recomendações, possibilitando melhores práticas no desenvolvimento de páginas da Web. A eMAG foi feita com base na WCAG, mostrada anteriormente. Nela exitem seis pontos de recomendações, que são marcação, comportamento, conteúdo, apresentação, multimídia e formulário [13].

\paragraph{1 - Marcação: }

O primeiro ponto, chamado de marcação, tem nove recomendações, que são, respeitar os padrões da Web, organizar o código HTML de forma lógica e semântica, utilizar corretamente os níveis de cabeçalho, ordenar de forma lógica e intuitiva a leitura e tabulação, fornecer âncoras para ir direto a um bloco de conteúdo, não utilizar tabelas para diagramação, separar links adjacentes, dividir as áreas de informação e não abrir novas instâncias sem a solicitação do usuário.

\paragraph{2 - Comportamento: }

O segundo ponto, chamado de comportamento, tem sete recomendações, que são disponibilizar todas as funções da página via teclado, garantir que os objetos programáveis sejam acessíveis, não criar páginas com atualização automática periódica, não utilizar redirecionamento automático de páginas, fornecer alternativa para modificar limite de tempo, não incluir situações com intermitência de tela e assegurar o controle do usuário sobre as alterações temporais do conteúdo.

\paragraph{3 - Conteúdo: }

O terceiro ponto, chamado de conteúdo, tem doze recomendações, que são identificar o idioma principal da página, informar mudança de idioma no conteúdo, oferecer um título descritivo e informativo à página, informar o usuário sobre sua localização na página, descrever links clara e sucintamente, fornecer alternativa em texto para as imagens do sítio, utilizar mapas de imagem de forma acessível, disponibilizar documentos em formatos acessíveis, em tabelas, utilizar títulos e resumos de forma apropriada, associar células de dados às células de cabeçalho, garantir a leitura e compreensão das informações e disponibilizar uma explicação para siglas, abreviaturas e palavras incomuns.

\paragraph{4 - Apresentação: }

No quarto ponto, chamado de apresentação ou design, tem quatro recomendações, que são oferecer contraste mínimo entre plano de fundo e primeiro plano, não utilizar apenas cor ou outras características sensoriais para diferenciar elementos, permitir redimensionamento sem perda de funcionalidade e possibilitar que o elemento com foco seja visualmente evidente.

\paragraph{5 - Multimídia: }

No quinto ponto, chamado de multimídia, tem cinco recomendações, que são fornecer alternativa para vídeo, fornecer alternativa para áudio, oferecer audiodescrição para vídeo pré-gravado, fornecer controle de áudio para som e fornecer controle de animação.

\paragraph{6 - Formulários: }

No sexto e ultimo ponto, chamado de formulários, tem oito recomendações, fornecer alternativa em texto para os botões de imagem de formulários, associar etiquetas aos seus campos, estabelecer uma ordem lógica de navegação, não provocar automaticamente alteração no contexto, fornecer instruções para entrada de dados, identificar e descrever erros de entrada de dados e confirmar o envio das informações, agrupar campos de formulário e fornecer estratégias de segurança específicas ao invés de CAPTCHA.

Além desses seis pontos com suas recomendações, a diretriz disponibilizou mais cinco elementos padronizados de acessibilidade digital ditos pelo Governo Federal, que são atalhos de teclado, primeira folha de contraste, barra de acessibilidade, apresentação do mapa do sítio e página de descrição com os recursos de acessibilidade.

\subsubsection{Conclusão das diretrizes}

Portanto, percebe-se que as diretrizes são bem extensas e abrange quase todas as formas de acessibilidade existente. Agora é preciso só que os programadores sigam cada princípio descrito, com a meta de fazer tal atividade em seus trechos de códigos, fazendo com que todas as pessoas consigam manipular os seus sites, sem haver nenhum tipo de discriminação por partes dos programadores e dos produtores de conteúdos. Nessa monografia, foi escolhido o uso da diretriz WCAG, já que ela é uma diretriz usada mundialmente por vários artigos nas análises dos sites estudados. A seguir, será mostrado trabalhos relacionados a acessibilidade de algumas plataformas, que usaram a WCAG nos seus estudos.

\subsection{Avaliação da acessibilidade de sites}
Neste subcapítulo são apresentados artigos sobre avaliação da acessibilidade de sites em diferentes campos.

\subsubsection{Sites educacionais}

No artigo [1] os autores avaliaram 107 sites educacionais brasileiros, através da existência de atalhos no teclado, de alto contraste, de barra de acessibilidade, de mapa de sitio e de paginas com descrições. Após a análise, concluíram um elevado descaso com a inclusão digital. O artigo [11] tinha como objetivo avaliar a acessibilidade de 91.421 vídeos de 113 universidades do mundo. Os resultados mostram, que embora apenas 17\% do total de vídeos tenham legendas associadas, o cumprimento deste critério de sucesso tem melhorado ao longo dos anos. Porém, não existe acessibilidade em relação à linguagem de sinais, à descrição de áudio ou à descrição de áudio estendida com vídeos.

No artigo [8], foi feita uma análise para usuários de baixa visão de 40 sites, sendo sites de 30 universidades da América Latina e dez sites dos mais visitados da região. O método utilizado foi o ``Testes de Triagem de Barreiras", proposto pelo Brajnik e pela WCAG, e a técnica consiste em priorizar os impactos das barreiras de acordo com o contexto aplicado, identificando muitas páginas web que violaram princípios da diretriz WCAG, totalizando 241 barreiras.

Além disso, no artigo [12] tinha como objetivo avaliar a acessibilidade de uma plataforma de aprendizagem, interrogando seus participantes. A análise foi feita através de alunos cegos, surdos e surdos-cegos, que utilizaram um protótipo de plataforma de aprendizagem acessível. Os pontos avaliados são texto alternativo, descrição de áudio, legendas, alto contraste, descrição longa e linguagem de sinais e transcrição, além da existência do perigo de convulsões. 

Por fim, no artigo [14] foi feito uma avaliação da acessibilidade de 110 sites de universidades estaduais e 69 sites de universidades privadas. Apenas dez sites de universidades estaduais e 4 de universidades privadas alcançaram bons níveis de acessibilidade, indicando baixa usabilidade, sendo assim pouco acessível para os deficientes visuais e auditivos. Os critérios avaliados foram conteúdo não textual, informações e relacionamentos, uso de cor, teclado, tempo ajustável, pausar, parar, ocultar, bloqueios de desvio, página intitulada, finalidade do link (no contexto), idioma da página, na entrada, etiquetas ou instruções, análise, nome, função, valor, contraste (mínimo), redimensionar texto, títulos e rótulos, contraste (aprimorado), teclado (sem exceção), finalidade do link (apenas link) e títulos de seção.

\subsubsection{Sites governamentais}

Alguns artigos se propõem a avaliar a acessibilidade em sites governamentais. Em [7], foi feito uma análise da acessibilidade dos sites governamentais da Arábia Saudita, investigando a existência dos seguintes pontos, boa informação de conteúdo não textual e seus relacionamentos, contraste, imagens de texto e vídeos de libras, atalhos de teclado, interrupções e uma boa sequência significativa. Segundo eles, os resultados da avaliação são promissores e indicam um aumento da conscientização para acessibilidade da web.

Já o artigo [9] investigou 22 sites de governo eletrônico móvel da Arábia Saudita analisando a existência de texto alternativo as imagens, alto contraste no texto, descrição dos link, formulários de fácil compreensão e textos que favoreçam o uso dos leitores de tela, e que tenham uma boa sequência significativa. O artigo concluiu que há problemas de usabilidade e acessibilidade que afetam o desempenho de sites governamentais.

Além desses, em [15], em que foi feita uma análise de 63 sites pertencentes a ministérios, departamentos e agências governamentais do Governo de Uganda, e em [17], que foram avaliados dez sites do governo líbio, seus respectivos autores usaram todas as diretrizes da WCAG e duas ferramentas de avaliação automática, a TAW e a AChecker. Eles chegaram a conclusão de que nem todos os sites atendem as diretrizes, tendo problemas de usabilidade e necessitando de uma grande melhoria na acessibilidade em sites de governos eletrônicos.

\subsubsection{Diretrizes analisadas}

Nesse subcapítulo, iremos mostrar quais diretrizes das WCAG foram usadas, pelos autores dos artigos para fazer as análises dos sites. Assim, a seguir, na tabela 1, será visto uma comparação dos artigos citados, individualmente.\\

Tabela 1 - Diretrizes da WCAG analisadas nos sites\\[-1cm]
\begin{center}
	\begin{longtable}{|c|c|c|c|}
		\hline
		Artigos & Autores & Ano & Diretrizes cumpridas\\
		\hline
		[1] & M. Campoverde-Molina,& 2020 & 1.2.3/1.2.5/1.2.7/1.4.3\\
			& S. Lujan-Mora e & & 1.4.5/1.4.6/1.4.9/2.1.1/\\
			& L. Valverde Gracia & & 2.1.2/2.1.3\\
		\hline
		[7] & H. S. Al-Khalifa, & 2016 & 1.2.6/1.3.2/1.4.3/1.4.5/\\
			& I. Baazeem e & & 1.4.6/1.4.9/2.1.1/2.1.2/\\
			& R. Alamer1 & & 2.1.3/2.2.4\\
		\hline
		[8] & P. Acosta-Vargas, & 2018 & TODAS\\
			& T. Acosta e & & \\
			& S. Luján-Mora & & \\
		\hline
		[9] & H. O. Al-Sakran e & 2021 & 1.2.8/1.2.9/1.3.2/1.4.3/\\
			& M. A, Alsudairi & & 1.4.5/1.4.6/1.4.9/4.1.2\\
		\hline
		[11] & T. Acosta, & 2020 & 1.1.1/1.2.1/1.2.2/1.2.3/\\
			 & P. Acosta Vargas, & & 1.2.4/1.2.5/1.2.6\\
			 & J. Zambrano Miranda e & & \\
			 & S. Luján Mora & & \\
		\hline
		[12] & C. Batanero-Ochaíta1, & 2017 & 1.2.2/1.2.3/1.2.4/\\
			 & L. De-Marcos, & & 1.2.6/1.2.8/1.4.3/\\
			 & L. Felipe Rivera, & & 2.3.1/2.3.2\\
			 & J. Ramón Hilera e & & \\
			 & S. Otón & & \\
		\hline
		[14] & Y. Akgül1n & 2020 & 1.1.1/1.3.1/1.4.1/1.4.3/\\
			 & & & 1.4.4/1.4.6/2.1.1/2.1.3/\\
			 & & & 2.2.1/2.2.2/2.4.1/2.4.2/\\
			 & & & 2.4.4/2.4.6/2.4.9/\\
			 & & & 2.4.10/3.1.1/3.2.2/3.3.2/\\
			 & & & 4.1.1/4.1.2\\
		\hline
		[15] & J. Nakatumba Nabende, & 2019 & TODAS\\
			 & B. Kanagwa, & &\\
			 & F. Nameere Kivunike e & &\\
			 & M. Tuape & &\\
		\hline
		[17] & N. A. Karaim e Y. Inal & 2017 & TODAS\\
		\hline
	\end{longtable}
\end{center}

Portanto, podemos ver que existe uma grande variedade de trabalhos sobre acessibilidade digital em todo o mundo, mostrando que as pesquisas estão bastante avançadas nesse ponto. Porém, como foi mostrado, individualmente, em cada pesquisa feita, que a acessibilidade digital para pessoas com deficiência visual e auditiva, tanto nos sites educacionais como nos sites governamentais, tem um percentual bastante pequeno. A seguir, será descrito as formas de deficiência visual que prejudica a usabilidade dos sites de notícias.

\subsection{Formas de deficiência visual}
A deficiência visual se divide em dois grupos com características e necessidades diferentes. Os dois grupos de pessoas são aqueles que apresentam baixa visão e aqueles com cegueira.

Cegueira é usado para identificar a condição de pessoas que apresentam total incapacidade de enxergar e também para aquelas com uma visão residual que, apesar de não ser a perda total da visão, possuem dificuldades ao realizar suas atividades diárias normalmente [20].

Assim, o cego é aquele que não utiliza a visão para a aprendizagem e que necessita de sistemas Braille ou de sistemas que verbalizam textos em computadores. Baixa visão ou visão subnormal é o termo usado para a pessoa que tem sua função visual comprometida, mas que usa ou é potencialmente capaz de usar a visão para executar tarefas. Portanto, o termo cegueira não é absoluto, pois reúne indivíduos com vários graus de visão residual. Ela não significa, necessariamente, total incapacidade para ver, mas prejuízo dessa aptidão a níveis incapacitantes para o exercício de tarefas rotineiras [19].

Além disso, as pessoas que possuem baixa visão tem uma condição na qual a sua visão não pode ser totalmente corrigida por óculos, interferindo em suas atividades diárias, assim como a leitura e a locomoção. A baixa visão pode ser causado pela degeneração macular, glaucoma, retinopatia diabética ou catarata. As pessoas com baixa visão necessitam do uso de óculos, lentes corretivas, lupas simples e/ou eletrônicas, além de caracteres ampliados e de uso de tecnologias assistivas como alto contraste e leitores de tela. Segundo a Organização Mundial da Saúde (OMS), em 1992, pessoas com baixa visão são aquelas que tem um bom comprometimento do funcionamento visual, mesmo após o tratamento ou correção refrativa padrão. Eles tem a acuidade visual de menos de 6/18 para percepção de luz, ou um campo visual de menos de dez graus do ponto de fixação, mas quem usa, ou é potencialmente capaz de usar, possuem dificuldades para o planejamento ou execução de algumas tarefas [18].

A cegueira adquirida causa uma ruptura nos padrões já constituídos de comunicação, mobilidade, trabalho, recreação e sentimentos, acerca de si próprio, tornando-se uma experiência inevitavelmente traumática. Existem dois tipos de cegueira adquirida: a cegueira súbita e a progressiva. A primeira pode ser divida em dois estágios: o choque imediato e a recuperação subsequente, esse estágio consiste em despersonalização seguida de depressão, parece ser uma defesa emergente contra a ameaça de dissolução do ego, os afetos são afastados para emergir, passo a passo, por um ego em pedaços. Nos casos de cegueira progressiva, o processo de despersonalização pode não ocorrer e a depressão não é severa, pois a fase de lamentação acontece antes da cegueira, onde a pessoa tem um tempo para digerir suas perdas. Por um lado esse tipo de cegueira pode facilitar o acesso e apoio antes da pessoa se tornar cega, contudo pode causar um estado de contínua ansiedade pela ameaça de perder a visão [21].

Assim, os usuários que possuem alguma deficiência visual são altamente prejudicados quando precisam acessar uma plataforma digital não acessível, pois eles ficam impossibilitados de usá-la pelo o fato delas não terem determinadas funcionalidades. Por exemplo, a falta de alto contraste, que deixe toda a tela em preto e branco (WCAG 1.4.3 Contraste / WCAG 1.4.6 Contraste (Melhorado)) e a falta de uma diferença de cores nas funcionalidades dos sites (WCAG 1.4.1 Utilização de cor), que afetam as pessoas com daltonismo. Outrossim, as pessoas que não conseguem ler determinadas palavras pelo tamanho da letra, isto é, pessoas que possuem baixa visão, elas precisam da funcionalidade de aumentar a fonte (WCAG 1.4.4 Redimensionar texto), com o objetivo de enxergarem tudo que o site propõe mostrar aos seus usuários.

Por fim, existem aquelas pessoas que possuem a cegueira total, isto é, são aqueles que tem a total incapacidade para ver o mundo exterior, necessitando de sites que descrevam em texto as imagens, os áudios e as mídias para os seus usuários (WCAG 1.1.1 Conteúdo não textual / WCAG 1.2.1 Conteúdo só de áudio e só de vídeo (pré gravado) / WCAG 1.4.5 Imagens de Texto) ou uma audiodescrição para as mídias existentes (WCAG 1.2.3 Audiodescrição ou alternativa em mídia (pré-gravada) / WCAG 1.2.5 Audiodescrição (pré-gravada) / 1.2.7 Audiodescrição Alargada (pré-gravada)). Além disso, precisam capazes de usarem os leitores de tela, que podem ser utilizados quando os sites possuem funcionalidades via teclados (WCAG 1.3.2 Sequência com significado / WCAG 2.1.1 Teclado / WCAG 2.1.2 Sem bloqueio do teclado / WCAG 2.1.3 Teclado (sem exceção) / WCAG 2.4.3 Ordem do foco), dentre outros exemplos de técnicas para acessibilidade web. Os tipos de deficiência visual estão descritos na tabela 2, e cada uma delas relacionadas com as diretrizes faladas anteriormente.\\

Tabela 2 - Deficiências visuais e as diretrizes da WCAG\\[-1cm]
\begin{center}
	\begin{longtable}{|c|c|}
		\hline
		Deficiências & Diretrizes\\
		\hline
		Daltonismo & WCAG 1.4.3 Contraste\\
		& WCAG 1.4.6 Contraste (Melhorado)\\
		& WCAG 1.4.1 Utilização de cor\\
		\hline
		Baixa visão & WCAG 1.4.4 Redimensionar texto\\
		\hline
		Totalmente cegos & WCAG 1.1.1 Conteúdo não textual\\
		& WCAG 1.2.1 Conteúdo são de áudio e são de vídeo (pré gravado)\\
		& WCAG 1.4.5 Imagens de Texto\\
		& WCAG 1.2.3 Audiodescrição ou alternativa em mídia (pré-gravada)\\
		& WCAG 1.2.5 Audiodescrição (pré-gravada)\\
		& WCAG 1.2.7 Audiodescrição Alargada (pré-gravada)\\
		\hline
		Os três casos & WCAG 1.3.2 Sequência com significado\\
		& WCAG 2.1.1 Teclado\\ 
		& WCAG 2.1.2 Sem bloqueio do teclado\\
		& WCAG 2.1.3 Teclado (sem exceção)\\
		& WCAG 2.4.3 Ordem do foco\\
		\hline
	\end{longtable}
\end{center}

Portanto, esses são os exemplos das deficiências visuais que afetam a usabilidade dos usuários ao estarem acessando as plataformas digitais. A seguir será falado sobre a diversidade da deficiência auditiva.

\subsection{Formas de deficiência auditiva}
A deficiência auditiva é considerada como a diferença existente entre o desempenho do indivíduo e a habilidade normal para a detecção sonora. A audição normal corresponde à habilidade para detecção de sons até 20 dB N.A (decibéis, nível de audição). A audição serve para o desenvolvimento e na manutenção da comunicação por meio da linguagem falada [22]. Dentre os exemplos de deficiência auditiva está a condutiva, a sensório neural, a mista e a central.

A condutiva é causada por um problema localizado no ouvido externo e/ou médio, que tem por função conduzir o som até o ouvido interno. Esta deficiência, em muitos casos, é reversível e geralmente não precisa de tratamento com aparelho auditivo, apenas cuidados médicos [23]. Ela ocorre quando há uma interferência na transmissão do som desde o conduto auditivo externo até a orelha interna [22]. A grande maioria das deficiências auditivas condutivas pode ser corrigida através de tratamento cirúrgico ou medicamentoso [24]. Esta deficiência pode ter várias causas, dentre elas, pode-se citar, os corpos estranhos no conduto auditivo externo, tampões de cera, otite externa e média, mal formação congênita do conduto auditivo, inflamação da membrana timpânica, perfuração do tímpano, obstrução da tuba auditiva, entre outros [22].

Já a sensório-neural é quando há uma impossibilidade de recepção do som por lesão das células ciliadas da orelha interna ou do nervo auditivo. Este tipo de deficiência auditiva é irreversível. A deficiência auditiva sensório-neural pode ser de origem hereditária como problemas da mãe no pré-natal tais como a rubéola, sífilis, herpes, toxoplasmose, alcoolismo, toxemia, diabetes, entre outros. Também podem ser causadas por traumas físicos, prematuridade, baixo peso ao nascimento, trauma de parto, meningite, encefalite, caxumba, sarampo, entre outros [22]. Assim acontece uma diminuição na capacidade de receber os sons que passam pelo ouvido externo e ouvido médio. A deficiência sensório-neural faz com que as pessoas escutem menos e também tenham maior dificuldade de perceber as diferenças entre os sons [23].

A mista é quando há uma alteração na condução do som até o órgão terminal sensorial associada à lesão do órgão sensorial ou do nervo auditivo. O audiograma mostra, geralmente, limiares de condução óssea abaixo dos níveis normais, embora com comprometimento menos intenso do que nos limiares de condução aérea [22]. Sendo assim, a mista apresenta características condutiva e sensório-neural, pois a lesão envolve duas ou as três partes da orelha [24].

Por fim, a central ou surdez central não é, necessariamente, acompanhado de diminuição da sensitividade auditiva, mas manifesta-se por diferentes graus de dificuldade na compreensão das informações sonoras [22]. Ela decorre de alterações nos mecanismos de processamento da informação sonora no tronco cerebral [24].

Os níveis de limiares utilizados para caracterizar os graus de severidade da deficiência auditiva são, audição normal, limiares entre 0 a 24 dB nível de audição. Deficiência auditiva leve, limiares entre 25 a 40 dB nível de audição. Deficiência auditiva moderna, limiares entre 41 e 70 dB nível de audição. Deficiência auditiva severa, limiares entre 71 e 90 dB nível de audição. Deficiência auditiva profunda, limiares acima de 90 dB [22].

Dentre os muitos instrumentos usados para comunicação não oral está a figura a linguagem dos sinais, criada por um monge beneditino francês, morador de um mosteiro onde imperava a lei do silêncio. Adotada há mais de cem anos, no Brasil é chamada de Libras (WCAG 1.2.3 Língua gestual (pré-gravada)). Além disso, também é usado as legendas como alternativas as mídias e áudios (WCAG 1.2.2 Legendas (pré-gravadas) / WCAG 1.2.4 Legendas (em direito) / WCAG 1.2.8 Alternativa em multimídia (pré-gravada)), dentre outros exemplos de técnicas para acessibilidade web. Assim esses são os exemplos das deficiências auditivas que afetam o acesso de forma tranquila dos sites, inclusive os jornalísticos. Os tipos de deficiência auditiva estão descritos na tabela 3, e cada uma delas relacionadas com as diretrizes faladas anteriormente.\\

Tabela 3 - Deficiências auditiva e as diretrizes da WCAG\\[-1cm]
\begin{center}
	\begin{longtable}{|c|c|}
		\hline
		Todos os casos & WCAG 1.2.3 Língua gestual (pré-gravada)\\
		& WCAG 1.2.2 Legendas (pré-gravadas)\\ 
		& WCAG 1.2.4 Legendas (em direito)\\
		& WCAG 1.2.8 Alternativa em multimídia (pré-gravada)\\
		\hline
	\end{longtable}
\end{center}

\subsection{Conclusão do referencial teórico}

Vimos nesse capítulo duas diretrizes que são usadas para avaliar os mais diversos sites. Foi feito um detalhamento das suas diretrizes e dos seus pontos principais. As diretrizes foram a eMAG e a WCAG, na qual, essa foi escolhida para fazer a avaliação das plataformas digitais de notícias do Brasil nessa monografia. Além disso, foi visto alguns exemplos de artigos de sites educacionais e sites governamentais, onde os seus autores usaram a WCAG para analisar e computar a acessibilidade de cada um deles. Por fim, foram mostradas as formas de deficiência visual e auditiva, relacionando-as com as diretrizes da WCAG. A seguir, será mostrado a metodologia que será aplicada nessa monografia. \\[11cm]

\section{MATERIAIS E MÉTODOS}

Nesse capítulo iremos mostrar os passos que foram seguidos para se cumprir o propósito final, que é analisar a acessibilidade dos sites de notícia do Brasil, para assim serem quantificados, percentualmente, seguindo determinado critério de avaliação, em relação a sua acessibilidade. Ao todo foram feitos seis passos para se realizar os objetivos dessa monografia, sendo especificados em 5 subseções. Nesse próximo subcapítulo, será mostrado os passos da metodologia, resumidamente, aplicada nesse trabalho. 

\subsection{Passos da metodologia}

Então, a metodologia, que será aplicada para que os objetivos sejam alcançados, 
será primeiro escolher os princípios e as especificações da diretriz WCAG que serão usadas nas análises dos sites, descrevendo e detalhando cada uma delas. 
Depois será mostrado a forma utilizada para fazer a computação dos sites de notícia que serão avaliados, usando as diretrizes escolhidas. 
A seguir será escolhido os dez sites de notícias do Brasil, seguindo um determinado critério, mostrando, rapidamente algumas informações relevantes da historia de cada site. 
O quarto passo da metodologia será realizar uma avaliação manual da acessibilidade de cada site de acordo com as especificações escolhidas da WCAG. Após isso, será coletado dados do levantamento feito a partir da avaliação manual. Por fim, será feito um comparativo e uma análise dos resultados da computação entre os sites. O passos da metodologia estão demostrados na figura 1.\\

Figura 1 - Fluxograma das etapas da metodologia\\[-0.7cm]
\begin{figure}[!ht]
	\centering
	% Define block styles
	\tikzstyle{block} = [rectangle, draw, fill=white, text width=5em, text centered, node distance=1.6cm]
	\tikzstyle{line} = [draw, -latex]
	\begin{tikzpicture}[node distance = 3cm, auto]
		% Place nodes
		\node [block] (WCAG) {Escolher os princípios da diretriz WCAG};
		\node [block, right of = WCAG, xshift=0.8cm] (computar) {Métodos para computar os sites};
		\node [block, right of = computar, xshift=0.8cm] (sites) {Selecionar os dez sites para avaliação};
		\node [block, right of = sites, xshift=0.8cm] (avaliacao) {Avaliar manualmente os sites};
		\node [block, right of = avaliacao, xshift=0.8cm] (coletar) {Coletar dados a partir da avaliação};
		\node [block, right of = coletar, xshift=0.8cm] (comparativo) {Análisar os resultados e comparar os sites};
		% Draw edges
		\path [line] (WCAG) -- (computar);
		\path [line] (computar)   -- (sites);
		\path [line] (sites)   -- (avaliacao);
		\path [line] (avaliacao)   -- (coletar);
		\path [line] (coletar)      -- (comparativo);
	\end{tikzpicture}
\end{figure}\\

Portanto, esse será o caminho percorrido por essa monografia. Assim, à seguir, será desenvolvido o primeiro passo que será o detalhamento dos princípios e das especificações da WCAG, relacionadas com a deficiência visual e com a deficiência auditiva. A WCAG não faz essa classificação, mas como o intuito do trabalho é avaliar especificamente para as deficiências visual e auditiva, foi feito essa análise das diretrizes, já que essas especificações serão usadas para avaliar as plataformas.

\subsection{Especificações da WCAG e a deficiência visual}

Existem algumas especificações da WCAG que estão ligadas diretamente com a deficiência visual. As especificações escolhidas estão descritas nesse subcapítulo e todas as foram tiradas diretamente da documentação da WCAG [10]. No quadro 5, logo abaixo, será mostrado as diretrizes que foram escolhidas.\\

Quadro 5 - Especificações da WCAG relacionadas com a deficiência visual\\[-1cm]
\begin{center}
	\begin{longtable}{|c|c|}
		\hline
		Diretrizes escolhidas & Nome das diretrizes\\
		\hline
		1.1.1 & Conteúdo não textual\\
		\hline
		1.2.1 & Conteúdo só de áudio e só de vídeo (pré-gravado)\\
		\hline
		1.2.3 & Audiodescrição ou alternativa em mídia (pré-gravada)\\
		\hline
		1.2.5 & Audiodescrição (pré-gravada)\\
		\hline
		1.2.7 & Audiodescrição alargada (pré-gravada)\\
		\hline
		1.3.2 & Sequência com significado\\
		\hline
		1.4.1 & Utilização da cor\\
		\hline
		1.4.3 & Contraste\\
		\hline
		1.4.4 & Redimensionar texto\\
		\hline
		1.4.5 & Imagem de texto\\
		\hline
		1.4.6 & Contraste (melhorado)\\
		\hline
		2.1.1 & Teclado\\
		\hline
		2.1.2 & Sem bloqueio do teclado\\
		\hline
		2.1.3 & Teclado (sem exceção)\\
		\hline
		2.4.3 & Ordem do foco\\
		\hline
	\end{longtable}
\end{center}

À seguir, será descrito, separadamente, cada diretriz escolhida e listada no quadro 5 que tem alguma relação com a deficiência visual.

\subsubsection{1.1.1 Conteúdo não textual}

Todo o conteúdo não textual que é exibido ao usuário tem uma alternativa textual que serve a um propósito equivalente, exceto para as situações indicadas abaixo (Nível A).\\

\hspace{.1\textwidth} %posiciona a minipage
\begin{minipage}{.85\textwidth}
	1) Controles, Entrada: Se o conteúdo não textual for um controle ou aceitar a entrada de dados por parte do usuário, então esse conteúdo não textual possui um nome que descreve a sua finalidade (Consultar a Diretriz 4.1 para requisitos adicionais de controles e conteúdo que aceitam entrada de dados por parte do usuário);\\
\end{minipage}

\hspace{.1\textwidth} %posiciona a minipage
\begin{minipage}{.85\textwidth}
	2) Mídias com base no tempo: Se o conteúdo não textual consiste em mídia baseada em tempo, então as alternativas textuais fornecem, no mínimo, uma identificação descritiva do conteúdo não textual (Consulte a Diretriz 1.2 para obter os requisitos adicionais para mídia);\\
\end{minipage}

\hspace{.1\textwidth} %posiciona a minipage
\begin{minipage}{.85\textwidth}
	3) Teste: Se o conteúdo não textual for um teste ou um exercício, que ficaria inválido se fosse apresentado em texto, então as alternativas textuais fornecem, no mínimo, uma identificação descritiva do conteúdo não textual;\\
\end{minipage}

\hspace{.1\textwidth} %posiciona a minipage
\begin{minipage}{.85\textwidth}
	4) Sensorial: Se a finalidade do conteúdo não textual for, essencialmente, criar uma experiência sensorial específica, então as alternativas textuais fornecem, no mínimo, uma identificação descritiva do conteúdo não textual;\\
\end{minipage}

\hspace{.1\textwidth} %posiciona a minipage
\begin{minipage}{.85\textwidth}
	5) CAPTCHA: Se a finalidade do conteúdo não textual for confirmar que o conteúdo está sendo acessado por uma pessoa e não por um computador, então devem ser fornecidas alternativas textuais que identificam e descrevem a finalidade do conteúdo não textual. Formas alternativas de CAPTCHA, que utilizam modos de saída para diferentes tipos de percepção sensorial, devem ser apresentadas para atender diferentes deficiências;\\
\end{minipage}

\hspace{.1\textwidth} %posiciona a minipage
\begin{minipage}{.85\textwidth}
	6) Decoração, Formatação, Invisível: Se o conteúdo não textual for meramente decorativo, se for utilizado apenas para formatação visual, ou se não for exibido aos usuários, então esse conteúdo não textual deve ser implementado de uma forma que possa ser ignorado pelas tecnologias assistivas.\\
\end{minipage}

\subsubsection{1.2.1 Conteúdo só de áudio e só de vídeo (pré-gravado)}

Para as mídias de apenas áudio pré-gravados e mídias de vídeo (sem áudio) pré-gravados as regras seguintes são verdadeiras, exceto quando o áudio ou o vídeo é uma mídia alternativa para o texto e está claramente identificado como tal: (Nível A).\\

\hspace{.1\textwidth} %posiciona a minipage
\begin{minipage}{.85\textwidth}
	1) Apenas áudio pré-gravado: É fornecida uma alternativa para mídia com base em tempo, que apresenta informação equivalente para o conteúdo composto por apenas áudio pré-gravado;\\
\end{minipage}

\hspace{.1\textwidth} %posiciona a minipage
\begin{minipage}{.85\textwidth}
	2) Apenas vídeo pré-gravado: É fornecida uma alternativa em mídia com base em tempo ou uma faixa de áudio que apresenta informação equivalente para o conteúdo apenas de vídeo pré-gravado.
\end{minipage}
\subsubsection{1.2.3 Audiodescrição ou alternativa em mídia (pré-gravada)}

Uma alternativa para mídia com base em tempo ou uma audiodescrição do conteúdo em vídeo pré-gravado é fornecida para mídia sincronizada, exceto quando a mídia é uma alternativa ao texto e for claramente identificada como tal. (Nível A)

\subsubsection{1.2.5 Audiodescrição (pré-gravada)}

É fornecida audiodescrição para todo o conteúdo de vídeo pré-gravado existente em mídia sincronizada. (Nível AA)

\subsubsection{1.2.7 Audiodescrição alargada (pré-gravada)}

Quando as pausas no áudio de primeiro plano forem insuficientes para permitir que as audiodescrições transmitam o sentido do vídeo, é fornecida uma audiodescrição estendida para todo o vídeo pré-gravado existente no conteúdo em mídia sincronizada. (Nível AAA)

\subsubsection{1.3.2 Sequência com significado}

Quando a sequência na qual o conteúdo é apresentado afeta o seu significado, uma sequência de leitura correta pode ser determinada por meio de código de programação. (Nível A)

\subsubsection{1.4.1 Utilização da cor}

A cor não é utilizada como o único meio visual de transmitir informações, indicar uma ação, pedir uma resposta ou distinguir um elemento visual. (Nível A). 

Este critério de sucesso aborda especificamente a percepção de cores. Outras formas de percepção são abordadas na Diretriz 1.3, incluindo o acesso às cores por meio de código de programação e a outra codificação da apresentação visual.

\subsubsection{1.4.3 Contraste}

A apresentação visual de texto e imagens de texto tem uma relação de contraste de, no mínimo, 4.5:1, exceto para o seguinte: (Nível AA)\\

\hspace{.1\textwidth} %posiciona a minipage
\begin{minipage}{.85\textwidth}
	1) Texto Ampliado: Texto em tamanho grande e as imagens compostas por texto em tamanho grande têm uma relação de contraste de, no mínimo, 3:1;\\
\end{minipage}

\hspace{.1\textwidth} %posiciona a minipage
\begin{minipage}{.85\textwidth}
	2) Texto em plano Secundário: O texto ou imagens de texto que fazem parte de um componente de interface de usuário inativo, que são meramente decorativos, que não estão visíveis para ninguém, ou que são parte de uma imagem que inclui outro conteúdo visual significativo, não têm requisito de contraste;\\
\end{minipage}

\hspace{.1\textwidth} %posiciona a minipage
\begin{minipage}{.85\textwidth}
	3) Logotipos: O texto que faz parte de um logotipo ou marca comercial não tem requisito de contraste.\\
\end{minipage}

\subsubsection{1.4.4 Redimensionar texto}

Exceto para legendas e imagens de texto, o texto pode ser redimensionado sem tecnologia assistiva até 200 por cento sem perder conteúdo ou funcionalidade. (Nível AA)

\subsubsection{1.4.5 Imagens de texto}

Se as tecnologias que estiverem sendo utilizadas puderem proporcionar a apresentação visual, é utilizado texto para transmitir informações em vez de imagens de texto, exceto para o seguinte: (Nível AA)\\

\hspace{.1\textwidth} %posiciona a minipage
\begin{minipage}{.85\textwidth}
	1) Personalizável: A imagem de texto pode ser visualmente personalizada de acordo com os requisitos do usuário;\\
\end{minipage}

\hspace{.1\textwidth} %posiciona a minipage
\begin{minipage}{.85\textwidth}
	2) Essencial: Uma determinada apresentação de texto é essencial para as informações que serão transmitidas.\\
\end{minipage}

Os logotipos (texto que faz parte de um logotipo ou marca comercial) são considerados essenciais.

\subsubsection{1.4.6 Contraste (melhorado)}

A apresentação visual do texto e imagens de texto tem uma relação de contraste de, no mínimo, 7:1, exceto para as seguintes situações: (Nível AAA)\\

\hspace{.1\textwidth} %posiciona a minipage
\begin{minipage}{.85\textwidth}
	1) Texto Ampliado: Texto em tamanho grande e as imagens compostas por texto em tamanho grande têm uma relação de contraste de, no mínimo, 4.5:1;\\
\end{minipage}

\hspace{.1\textwidth} %posiciona a minipage
\begin{minipage}{.85\textwidth}
	2) Texto em plano Secundário: O texto ou as imagens de texto que fazem parte de um componente de interface de usuário inativo, que são meramente decorativos, que não estão visíveis para ninguém, ou que fazem parte de uma imagem que inclui outro conteúdo visual significativo, não têm requisito de contraste;\\
\end{minipage}

\hspace{.1\textwidth} %posiciona a minipage
\begin{minipage}{.85\textwidth}
	3) Logotipos: O texto que faz parte de um logotipo ou marca comercial não tem requisito de contraste mínimo.\\
\end{minipage}

\subsubsection{2.1.1 Teclado}

Toda a funcionalidade do conteúdo é operável através de uma interface de teclado sem requerer temporizações específicas para digitação individual, exceto quando a função subjacente requer entrada de dados que dependa da cadeia de movimento do usuário e não apenas dos pontos finais. (Nível A)

Esta exceção diz respeito à função subjacente, não à técnica de entrada de dados. Por exemplo, se utilizar escrita manual para introduzir texto, a técnica de entrada de dados (escrita manual) requer entrada de dados dependente de caminho, mas a função subjacente (entrada de texto) não.

Isto não proíbe, e não deve desencorajar, a entrada de dados através do mouse ou outros métodos de entrada de dados em conjunto à operação com o teclado.

\subsubsection{2.1.2 Sem bloqueio do teclado}

Se o foco do teclado puder ser movido para um componente da página utilizando uma interface de teclado, então o foco pode ser retirado desse componente utilizando apenas uma interface de teclado e, se for necessário mais do que as setas do cursor ou tabulação ou outros métodos de saída normalmente utilizados, o usuário deve ser informado sobre o método para retirar o foco. (Nível A)

Uma vez que qualquer conteúdo que não cumpra este critério de sucesso pode interferir com a capacidade de um usuário usar toda a página, todo o conteúdo da página web (quer seja utilizado para cumprir outros critérios de sucesso ou não) tem que cumprir este critério de sucesso.

\subsubsection{2.1.3 Teclado (sem exceção)}

Toda a funcionalidade do conteúdo é operável através de uma interface de teclado sem requerer temporizações específicas para digitação individual. (Nível AAA)

\subsubsection{2.4.3 Ordem do foco}

Se uma página web puder ser navegada de forma sequencial e as sequências de navegação afetarem o significado ou a operação, os componentes que podem ser focados recebem o foco em uma ordem que preserva o significado e a operabilidade. (Nível A)

Esses são as 15 diretrizes escolhidas relacionadas com a deficiência visual que será usado na análise dos sites de notícia. A seguir, será falado as diretrizes relacionadas com a deficiência auditiva.

\subsection{Especificações da WCAG e a deficiência auditiva}

Existem algumas especificações  da WCAG que estão ligadas diretamente com a deficiência auditiva. As especificações escolhidas estão descritas nesse subcapítulo e todas as foram tiradas diretamente da documentação da WCAG [10]. No quadro 6, logo abaixo, será mostrado as diretrizes escolhidas.\\

Quadro 6 - Especificações da WCAG relacionadas com a deficiência auditiva\\[-1cm]
\begin{center}
	\begin{longtable}{|c|c|}
		\hline
		Diretrizes escolhidas & Nome das diretrizes\\
		\hline
		1.2.2 & Legendas (pré-gravadas)\\
		\hline
		1.2.4 & Legendas (em direto)\\
		\hline
		1.2.6 & Língua Gestual (pré-gravada)\\
		\hline
		1.2.8 & Alternativa em Multimídia (pré-gravada)\\
		\hline
	\end{longtable}
\end{center}

À seguir, será descrito, separadamente, cada diretriz escolhida e listada no quadro 6 que tem alguma relação com a deficiência auditiva.

\subsubsection{1.2.2 Legendas (pré-gravadas)}

São fornecidas legendas para todo conteúdo de áudio pré-gravado em mídia sincronizada, exceto quando a mídia for uma alternativa para texto e for claramente identificada como tal. (Nível A)

\subsubsection{1.2.4 Legendas (em direto)}

São fornecidas legendas para todo o conteúdo do áudio ao vivo existente em mídia sincronizada. (Nível AA)

\subsubsection{1.2.6 Língua gestual (pré-gravada)}

É fornecida interpretação em língua de sinais para todo o conteúdo de áudio pré-gravado existente em um conteúdo em mídia sincronizada. (Nível AAA)

\subsubsection{1.2.8 Alternativa em multimídia (pré-gravada)}

É fornecida uma alternativa para mídia com base em tempo para a todo o conteúdo existente em mídia sincronizada pré-gravada e para a todo o conteúdo multimídia composto por apenas vídeo pré-gravado. (Nível AAA)

Esses são as 4 diretrizes escolhidas relacionadas com a deficiência auditiva que será usado na análise dos sites de notícia. Após isso, foi feito a escolha do método para mensurar a acessibilidade de cada site de notícia em relação às diretrizes pontuadas anteriormente.

\subsection{Método para quantificar a acessibilidade os sites}

Considerando as 19 diretrizes escolhidas no capítulo anterior, sendo 15 ligadas com a deficiência visual e mais 4 ligadas com a deficiência auditiva, será mostrado agora a forma que esse trabalho irá usar para quantificar a acessibilidade dos dez site de notícias que serão detalhados no próximo capítulo.

Como foi visto, no começo da monografia, no capítulo 2, chamado de referencial teórico e, dentro dela, no subcapítulo 2.2, chamada de avaliação da acessibilidade de sites, existem algumas formas para quantificá-los. Porém, iremos usar um método proposto pelo o próprio autor desse trabalho, cada diretriz possui um quantificador de importância que pode ser ``A", ``AA" ou ``AAA", sendo que, o ``AAA" é o mais importante que o site tenha em relação a diretriz de valor ``AA", que por sua vez é mais importante que a diretriz ``A". Desse modo, será definido pesos, em que o ``AAA" terá peso 3, o ``AA" terá peso 2 e o ``A" terá peso dois, como está sendo mostrado na tabela abaixo com as diretrizes, seus quantificadores e seus pesos.\\

Tabela 4 - Diretrizes escolhidas e seus pesos\\[-1cm]
\begin{center}
	\begin{longtable}{|c|c|c|}
		\hline
		Diretriz & Quantificador & Peso\\
		\hline
		1.1.1 Conteúdo não textual & A & 1\\
		\hline
		1.2.1 Conteúdo só de áudio e & A & 1\\
		só de vídeo (pré-gravado) & & \\
		\hline
		1.2.2 Legendas (pré-gravadas) & A & 1\\
		\hline
		1.2.3 Audiodescrição ou alternativa & A & 1\\
		em mídia (pré-gravada) & & \\
		\hline
		1.2.4 Legendas (em direto) & AA & 2\\
		\hline
		1.2.5 Audiodescrição (pré-gravada) & AA & 2\\
		\hline
		1.2.6 Língua Gestual (pré-gravada) & AAA & 3\\
		\hline
		1.2.7 Audiodescrição Alargada (pré-gravada) & AAA & 3\\
		\hline
		1.2.8 Alternativa em Multimídia (pré-gravada) & AAA & 3\\
		\hline
		1.3.2 Sequência com Significado & A & 1\\
		\hline
		1.4.1 Utilização da Cor & A & 1\\
		\hline
		1.4.3 Contraste & AA & 2\\
		\hline
		1.4.4 Redimensionar texto & AA & 2\\
		\hline
		1.4.5 Imagens de texto & AA & 2\\
		\hline
		1.4.6 Contraste (Melhorado) & AAA & 3\\
		\hline
		2.1.1 Teclado & A & 1\\
		\hline
		2.1.2 Sem Bloqueio do Teclado & A & 1\\
		\hline
		2.1.3 Teclado (Sem Exceção) & AAA & 3\\
		\hline
		2.4.3 Ordem do Foco & A & 1\\
		\hline
		TOTAL & - & 34\\
		\hline
	\end{longtable}
\end{center}

Portanto, no total de 19 diretrizes escolhidas se obtêm um total de 34 pontos. Então ao fazer o seguinte calculo 100/34=2.941176471..., se obtém aproximadamente 2.94\% para cada ponto cumprido. Dessa forma, as diretrizes com quantificadores ``A" que forem cumpridas valerão 2.94\%, aquelas com quantificadores ``AA" que forem seguidas valerão 5.88\%, e, por fim, as diretrizes com quantificadores ``AAA" que forem executadas valerão 8.82\%. Sendo assim, caso os sites cumpram os 34 pontos, eles serão 100\% acessível para, em relação à essas 19 diretrizes, para os deficientes auditivos e visuais. Logo após, foram escolhidos os sites de notícia seguindo um determinado critério, mostrado a seguir.

\subsection{Escolha dos sites}

Essa pesquisa escolheu os sites de notícia pelo fato deles terem um papel bastante importante para a sociedade atual, já que eles tem o compromisso de informar para cada pessoa, que as acessa, o que está acontecendo no Brasil e no mundo.

Os sites foram escolhidos através do número de visitas na compilação entre as ferramentas Alexa, SimilarWeb e SemRush [16] no ano de 2020, e todos os dados dos sites foram tirados dessa compilação. Os sites a seguir estão listados na ordem crescente de acessos, Clicrbs, Yahoo! Notícias, IG, MSN Brasil, Estadão, R7 Notícias, Folha De São Paulo, Terra Notícias, Uol Notícias e G1.

\paragraph{10 - Clicrbs: }

Em décimo lugar, o Clicrbs é um portal de notícias que atende em especifico a região do Rio Grande do Sul e Santa Catariana. Ele recebe quase 20 milhões de visitas por mês, estando assim na décima posição. O site possui parceria com Globo devido a sua filiação da RBS TV com 18 emissoras de TV e 8 emissoras de rádio cobrindo a região sul do país. O grupo também é dono do jornal impresso e site Zero hora. Além de mais 300 blogs em sua plataforma [16].

O clicRBS é um portal do Grupo RBS que oferece conteúdo online para o estado do Rio Grande do Sul. Integram o clicRBS os sites dos veículos de mídia impressa e eletrônica do Grupo RBS e canais próprios de notícias e serviços, conteúdo produzido por jornalistas e webdesigners. É possível acessar o clicRBS através da Globo.com no qual está hospedado. No clicRBS, o usuário recebe notícias atualizadas durante todo o dia, tem acesso aos conteúdos dos jornais da RBS e à programação ao vivo de rádios. O portal concentra também a maior parte das ações de interatividade dos veículos e oferece canais para as manifestações de opinião dos usuários de todas as mídias do grupo [25].

\paragraph{9 - Yahoo: }

Em nono lugar está o Yahoo! Notícias Brasil. Ele é um site informativo que mantem o seu corpo jornalísticos de notícias em dia, principalmente em relação a notícias do mundo, pois ele é alimentado pela rede internacional do Yahoo [16].

O Yahoo! foi criado por Jerry Yang e David Filo em 1994, nos Estados Unidos. Estudantes da Universidade Stanford, os fundadores da companhia desenvolveram a plataforma para catalogar sites interessantes na internet. Apesar de ter nascido apenas para uso próprio, rapidamente o Yahoo! ganhou popularidade dentro e fora da universidade. Atualmente, o Yahoo! é mundialmente conhecido por seus produtos e serviços de internet para consumidores e empresas. O portal de comunicação conta com motor de busca, e-mail, messenger, grupos, notícias, entre outros. Além disso, a companhia atua com hospedagem de sites, serviços de gestão e publicidade online [26].

\paragraph{8 - IG: }

Já em oitavo lugar está o IG, que é um dos portais que mais se reinventa na internet brasileira. Mudanças constantes e parcerias de peso tem reerguido esse grande portal que antes provia acesso internet na era discada. Atualmente, eles esperam crescer muito mais, apostando no agregador de blogs “iGlr“. O Site receberá visitas de todos os blogs parceiros [16].

Internet Group é um provedor de acesso à Internet brasileiro de banda larga e de acesso discado à Internet adquirido em 2004  pelos grupos GP Investimentos (Telemar) e Brasil Telecom e fundido aos portais iBest e BrTurbo, que já eram de propriedade da empresa de telefonia. Em 2010, o iG é adquirido pela Oi após a venda da Brasil Telecom. Em 18 de abril de 2012 a empresa portuguesa Ongoing anunciou a compra do portal. O abalo no mercado foi imediato, apesar de não ter sido o primeiro provedor gratuito brasileiro, atingindo grande popularidade num tempo em que a banda larga era pouco comum. O portal também estava bem posicionado, reunia um número crescente de sites competentes e criativos, liderados pelo Último Segundo, formando um conjunto de destaque quando a política de portais ainda buscava amadurecimento no Brasil [27].

\paragraph{7 - MSN Brasil: }

Em sétimo lugar temos o MSN Brasil, que é o site de notícias após fechar o hotmail. Isso foi o que impulsionou esse portal que mostra ser umas das fontes mais relevantes em notícias de última hora na internet [10].

O portal MSN, originalmente conhecido como The Microsoft Network, foi fundado em 1995 pela Microsoft. Criado para os usuários do Windows 95, o site surgiu como uma rede de serviços online. A partir de 1997, os conteúdos do portal passaram a ser distribuídos para todos os assinantes. Por vários anos, o MSN foi conhecido pelo seu serviço de mensagens instantâneas, o MSN Messenger. Nascido em 1999, o programa foi fundido com o Windows Messenger, dando origem ao Windows Live Messenger. O serviço foi descontinuado em 2013, depois de ter marcado a vida dos usuários da internet dos anos 2000. Atualmente, o MSN oferece e-mail, pesquisa, grupos, fóruns de discussão, entre outros, inclusive notícias. A plataforma conta com cerca de 330 milhões de usuários e está presente em, aproximadamente, 100 países. Os principais mercados do portal são os Estados Unidos, México, Japão e  Brasil, onde está inserido o MSN Brasil [28].

\paragraph{6 - Estadão: }

Na sexta posição está o portal de notícias do jornal O Estado de S. Paulo, que oferece notícias com imparcialidade e seriedade. Grandes colunistas renomados emitem sua opinião nas colunas e nos blogs do portal. É o lugar perfeito para quem esta habituado a ler jornais impressos, pois com advento da web e da leitura online, nascem todos os dias jornalistas preparados exclusivamente para esse meio. O portal recebe mais de 10 milhões de visitas por mês [10].

No dia 20 de fevereiro de 1995, o Grupo Estado entrava na Internet. O serviço noticioso foi inicialmente hospedado no site WordNews, nos Estados Unidos, porque naqueles primeiros dias de 1995 ainda não havia como registrar os domínios, os endereços do site que são digitados nos navegadores, de sites brasileiros. No NetEstado os leitores passaram a ter os conteúdos da edição impressa do jornal acrescidos de conteúdos especiais que o novo formato comportava. Em 1996 a Agência Estado implantaria um pioneiro sistema editorial que permitia a publicação de notícias com rapidez. A estreia do publicador aconteceu na Olimpíada de Atlanta, permitindo a transmissão online inovadora dos acontecimentos esportivos [29]. Em 1998, com a Copa do Mundo da França, pela primeira vez seria adotado o endereço estadao.com.br. No mesmo ano, as eleições presidenciais também contariam com o novo modelo de cobertura digital. No ano 2000, o portal Estadão.com.br estrearia o novo modelo que seria aprimorado até os dias atuais, com a incorporações das novas tecnologias, linguagem, ferramentas e outras inovações [29].

\paragraph{5 - R7: }

Em quinto lugar temos 0 R7.com, que segue o padrão do estabelecido pelo G1. Ele teve sua estreia em 27 de setembro de 2009 com grande quantidade de jornalistas vindo do UOL, G1, Folha do Estado de SP, Terra, Abril, Reuters. Hoje, ele já é o 5° portal mais acessado do Brasil, porém esse numero só foi alcançado graças as ultimas parcerias feitas com outros sites e blogs, ao qual suas audiência são somadas. O R7 fechou parceria com 5 dos 10 blogs mais acessados do Brasil e ele pertence a Central Record de Comunicação [10].

R7 ou R7.com é um portal de internet brasileiro criado em 2009 que atualmente pertence ao Grupo Record. Em 2017, o R7 anunciou que tornou-se o quinto maior da América Latina. Seu conteúdo é produzido com o apoio das estruturas da RecordTV, da Record News e também de suas filiadas e afiliadas, que produzem matérias através de páginas regionais. No R7 foi feito com investimento estimado em 300 milhões de reais, e na data de estreia contava com uma equipe de 300 profissionais [30].

\paragraph{4 - Folha de São Paulo: }

Já na quarta colocação está a Folha de São Paulo, que é uma versão online do jornal impresso de maior credibilidade do Brasil. Ela está entre as melhores fontes de notícias da atualidade. Com furos jornalísticos, entrevistas e notícias bombásticas, ele se mantem na liderança impressa das principais cidades do Brasil, talvez por isso o grande market share conquistado no mundo online. Se você deseja notícia de verdade, sem ligações com o entretenimento empregada por outros portais, esse é o lugar certo [10].

A história da Folha começa em 1921, com a criação do jornal ``Folha da Noite". Em julho de 1925, é criado o jornal ``Folha da Manhã", edição matutina da ``Folha da Noite". A ``Folha da Tarde" é fundada após 24 anos. Em 1º de janeiro de 1960, os três títulos da empresa se fundem e surge o jornal Folha de S.Paulo [31]. E em 1995 começa a funcionar o Centro Tecnológico Gráfico-Folha, em Tamboré. O jornal passa a circular com a maioria das páginas coloridas. Também nesse ano, a empresa lança a FolhaWeb, primeiro site de notícias em tempo real [32].

\paragraph{3 - Terra: }

O terceiro site jornalístico mais acessado no Brasil é o portal Terra, criado em 1999, caiu muito em audiência nos últimos anos no Brasil, mas com a nova repaginação no layout espera crescer novamente oferecendo o melhor das notícias, entretenimento e esportes, o tri-pé mais cobiçado da internet brasileira [10]. Terra é uma empresa brasileira de internet pertencente ao grupo espanhol Telefónica, um dos maiores conglomerados de telecomunicações fixas e móveis do mundo. Lançado no Brasil após aquisição do até então portal gaúcho de conteúdo Zaz (ex-Nutecnet). Além de atuar como portal de notícias, atualmente a companhia também trabalha com aplicativos de celular, publicidade e outros serviços digitais. Em agosto de 2017, o portal Terra passou a funcionar somente no Brasil, encerrando suas atividades em parte da América Latina, Espanha e EUA [33].

\paragraph{2 - Uol Notícias: }

Na vice colocação está o Uol Notícias. Ele é um portal conhecido por ter experiência e tamanho para oferecer notícias de qualidade e com credibilidade. Presume-se que o portal em conjunto com seus parceiros receba em torno de 57 milhões de visitas por mês. A Folha e a Band que estão nesse ranking também possui parcerias com eles [10]. Universo Online vai ao ar em 1996, como o primeiro portal de notícias do Brasil, com serviço de Bate-papo, edição diária da Folha de S. Paulo, arquivos da Folha com cerca de 250 mil textos, reportagens do The New York Times, traduzidas para o português, folha da tarde e notícias populares, classificados, roteiros, saúde e a revista IstoÉ [34].

\paragraph{1 - G1: }

E em primeiro lugar está o G1, que é disparado o maior site de notícias do Brasil, e vê de longe os concorrentes. Desde 2006 mantem o padrão Globo de jornalismo e com conteúdos multimídia vem tirando proveito da internet sobre os tradicionais meios de comunicação. Ele é um dos únicos portais que não necessita dos “números adicionais”, ou seja, aqueles sites parceiros que juntam sua audiência ao portal se tornando um só perante ao ibope e ao Comscore. Segundo dados do final de 2013, o portal da Globo recebe cerca de 49 milhões de acessos por mês [16].

\subsection{Análise manual e resultados finais}

A última fase da pesquisa, para que o intuito da pesquisa seja alcançada, é fazer uma avaliação manual dos dez sites com o objetivo de averiguar a acessibilidade de cada plataforma. Essa análise será feita usando as 19 diretrizes com suas especificações, relacionadas com a deficiência auditiva e visual, escolhidas anteriormente. Após isso, será feito uma compilação dos resultados realizados a partir da análise feita. Tudo isso para fazer um comparativo dos sites, gerando uma computação do grau de acessibilidade das plataformas digitais. A seguir será mostrado os resultados da análise da acessibilidade dos dez sites de notícia\\[8cm]

\section{RESULTADOS}

Nesse capítulo será analisado a acessibilidade de dez sites de notícias existentes no Brasil para pessoas com deficiência visual e auditiva, que será mostrado o resultado numa tabela geral das plataformas digitais, com o objetivo de averiguar o nível de acessibilidade dos dez sites.\\

Tabela 5 - Diretrizes cumpridas nos sites\\[-1cm]
\begin{center}
	\begin{longtable}{|c|c|c|c|c|}
		\hline
		Diretriz & Quantificador & Peso & Pesos & Diretrizes\\
		& & & cumpridos & cumpridas\\
		\hline
		1.1.1 Conteúdo não textual & A & 1 & 0 & \\
		\hline
		1.2.1 Conteúdo só de áudio e & A & 1 & 0 & \\
		só de vídeo (pré-gravado) & & & & \\
		\hline
		1.2.2 Legendas (pré-gravadas) & A & 1 & 0 & \\
		\hline
		1.2.3 Audiodescrição & A & 1 & 0 & \\
		ou alternativa em & & & & \\
		mídia (pré-gravada) & & & & \\
		\hline
		1.2.4 Legendas (em direto) & AA & 2 & 0 & \\
		\hline
		1.2.5 Audiodescrição & AA & 2 & 0 & \\
		(pré-gravada) & & & & \\
		\hline
		1.2.6 Língua Gestual & AAA & 3 & 0 & \\
		(pré-gravada) & AAA & 3 & & \\
		\hline
		1.2.7 Audiodescrição & AAA & 3 & 0 & \\
		Alargada (pré-gravada) & & & & \\
		\hline
		1.2.8 Alternativa em & AAA & 3 & 0 & \\
		Multimídia (pré-gravada) & & & & \\
		\hline
		1.3.2 Sequência com Significado & A & 1 & 1 & \\
		\hline
		1.4.1 Utilização da Cor & A & 1 & 1 &  \\
		\hline
		1.4.3 Contraste & AA & 2 & 0 & \\
		\hline
		1.4.4 Redimensionar texto & AA & 2 & 0 & \\
		\hline
		1.4.5 Imagens de texto & AA & 2 & 0 & \\
		\hline
		1.4.6 Contraste (Melhorado) & AAA & 3 & 3 & \\
		\hline
		2.1.1 Teclado & A & 1 & 0 & \\
		\hline
		2.1.2 Sem Bloqueio do Teclado & A & 1 & 0 & \\
		\hline
		2.1.3 Teclado (Sem Exceção) & AAA & 3 & 0 & \\
		\hline
		2.4.3 Ordem do Foco & A & 1 & 1 & \\
		\hline
		TOTAL & - & 34 &  & \\
		\hline
	\end{longtable}
\end{center}

\section{CONCLUSÃO}

Portanto, foi visto nessa monografia que 
\end{titlepage}

\begin{titlepage}
\section*{REFERÊNCIAS}
\addcontentsline{toc}{section}{Referências}
\hspace{-0.05\textwidth}
\begin{minipage}{1\textwidth}
[1] M. Campoverde-Molina, S. Lujan-Mora e L. Valverde Gracia, ``Empirical Studies on Web Accessibility of Educational Websites: A Systematic Literature Review" em IEEEAcess, Abril de 2020.
\end{minipage}\\[0.5cm] [2] D. Luís Arenhardt, T. Stefanel Franchi e S. Medianeira Flores Costa, M. Zampieri Grohmann, ``Acessibilidade digital: Uma análise em portais de Instituições Federais de Educação do Brasil" em Education Policy Analysis Archives, Abril de 2017.\\[0.5cm] [3] P. Acosta-Vargas, T. Acosta e S. Luján-Mora, ``Challenges to Assess Accessibility in Higher A Comparative Study of Latin America Universities" em IEEEAcess, Março de 2018.\\[0.5cm] [4] E. M. Molanes-López, A. Rodriguez-Ascaso, E. Letón e J. Pérez-Martín, ``Assessment of Video Accessibility by Students of a MOOC on Digital Materials for All a MOOC on Digital Materials for All" em IEEEAcess, Maio de 2021.\\[0.5cm] [5] A. Ismail e K. S. Kuppusamy, ``Web accessibility investigation and identification of major issues of higher education websites with statistical measures - A case study of college websites" em Journal of King Saud University – Computer and Information Sciences, Março de 2019.\\[0.5cm] [6] A. Ismail e K. S. Kuppusamy, ``Accessibility of Indian universities homepages An exploratory study" em Journal of King Saud University – Computer and Information Sciences, Junho de 2016.\\[0.5cm] [7] H. S. Al-Khalifa, I. Baazeem e R. Alamer1, ``Revisiting the accessibility of Saudi Arabia government websites" em Universal Access in the Information Society, Setembro de 2016.\\[0.5cm] [8] P. Acosta-Vargas, T. Acosta e S. Luján-Mora, ``A Heuristic Method to Evaluate Web Accessibility for Users With Low Vision" em IEEEAcess, Março de 2018.\\[0.5cm] [9] H. O. Al-Sakran e M. A, Alsudairi ``Usability and Accessibility Assessment of Saudi Arabia Mobile E-Government Websites" em IEEEAcess, Março de 2021.\\[0.5cm] [10] B. Caldwell, M. Cooper, L. Guarino Reid, G. Vanderheiden, W. Chisholm, J. Slatin e J. White, Consórcio W3C(2008), ``Diretrizes de Acessibilidade para Conteúdo Web (WCAG)2.0", Acesso em: 11 jun. 2021. [Online]. Disponível: https://www.w3.org/Translations/WCAG20-pt-PT/WCAG20-pt-PT-20141024/\\[0.5cm] [11] T. Acosta, P. Acosta Vargas, J. Zambrano Miranda e S. Luján Mora, ``Web Accessibility Evaluation of Videos Published on YouTube by Worldwide Top-Ranking Universities" em IEEEAcess, Junho de 2020.\\[0.5cm] [12] C. Batanero-Ochaíta1, L. De-Marcos, L. Felipe Rivera, J. Holvikivi, J. Ramón Hilera e S. Otón, ``Improving Accessibility in Online Education Comparative Analysis of Attitudes of Blind and Deaf Students Towards an Adapted Learning Platform" em IEEEAcess, 2017.\\[0.5cm] [13] B. Poletto Salton, J. Poletto Almeida, A. Rocha Façanha, A. Luiz Rezende, A. Poletto Sonza, Â. Guimarães, F. Zap, G. Samuel do Nascimento, J. Fiore de Oliveira Junior, J. Marques Carvalho da Silva, M. Antônio de Queiroz, M. Vinícius Bennett Ferreira, M. Covolan Rosito, R. Ferraz, R. Busatto, R. Moro, U. Paterno e W. Fernandes, A. Luís Porto Costa, E. Carniel, L. Nervis, J. Pilotti e L. Siqueira Lima, A. Pimenta Freire, C. Scarton, D. Roger Ramos Freitas, E. Marques Bento, J. Fernandes, L. Dell Anhol Almeida, M. Cecília Calani Baranauskas, T. Prado de Campos, V. Figueredo de Santana, A. E. Dourado Salina Gabriel, C. Gonçalves do Bomfim, E. Santos Martins Leite, F. Hoffmann Lobato e H. Gomes Mesquita, ``eMAG - Modelo de Acessibilidade em Governo Eletrônico", Acesso em: 29 jul. 2021. [Online]. Disponível: http://emag.governoeletronico.gov.br/\\[0.5cm] [14] Y. Akgül1n, ``Accessibility, usability, quality performance, and readability evaluation of university websites of Turkey: a comparative study of state and private universities" em Springer, Abril de 2020.\\[0.5cm] [15] J. Nakatumba Nabende, B. Kanagwa, F. Nameere Kivunike e M. Tuape, ``Evaluation of accessibility standards on Ugandan e-government websites" em Electronic Government an International Journal, Junho de 2019.\\[0.5cm] [16] A. S. Lucas, Top 10 maiores sites de notícias do Brasil", Acesso em: 03 ago. 2021. [Online]. Disponível: https://top10mais.org/sites-de-noticias/\\[0.5cm] [17] N. A. Karaim e Y. Inal, ``Usability and accessibility evaluation of Libyan government websites" em Department of Information Systems Engineering, Graduate
School of Natural and Applied Science, Atilim University, Ankara, Turkey e em Department of Information Systems Engineering, Faculty of Engineering, Atilim University, Ankara, Turkey, Outubro de 2017.\\[0.5cm] [18] S. Naipal1, N. Rampersad, ``A review of visual impairment" em University of KwaZulu-Natal, South Africa, Jan de 2018.\\[0.5cm] [19] A. de Castro Roma, ``Breve histórico do processo cultural e educativo dos deficientes visuais no Brasil" em Revista Ciência Contemporânea, Dez de 2018.\\[0.5cm] [20] Mundo Educação, ``Cegueira", Acesso em: 03 ago, 2021. [Online]. Disponível: https://mund oeducacao.uol.com.br/doencas/cegueira.htm\\[0.5cm] [21] S. Tamires Almeida e V. Filipe Araújo, ``Diferenças experienciais entre pessoas com cegueira congênita e adquirida: uma breve apreciação", em Revista Interfaces: Saúde, Humanas e Tecnologia, Ano 1, v. 1, n.3 e em Faculdade Leão Sampaio, Jun de 2013.\\[0.5cm] [22] Fio Cruz, ``Deficiência Auditiva", Acesso em: 13 ago, 2021. [Online]. Disponível: http://www .fiocruz.br/biosseguranca/Bis/infantil/deficiencia-auditiva.htm\\[0.5cm] [23] A. L. Campos Barbosa, J. J. Ferreira Reis Sampaio e N. Alves Marques, ``A educação do deficiente auditivo: um olhar sobre a diferença" em UNIT-SE, Maio de 2019\\[0.5cm] [24] F. V. Braga Junior e S. Andrade de Paula Bedaque, ``Deficiência auditiva e o atendimento educacional especializado" em SECADI e em Universidade Federal Rural do Semi-ÁridoReitor, em 2015\\[0.5cm] [25] Wikipédia - A enciclopédia livre, ``clicRBS", Acesso em: 19 ago, 2021. [Online]. Disponível: https://pt.wikipedia.org/wiki/ClicRBS\\[0.5cm] [26] Canaltech, ``Yahoo", Acesso em: 19 ago, 2021. [Online]. Disponível: https://canaltech.com. br/empresa/yahoo/\\[0.5cm] [27] Canaltech, ``Internet Group", Acesso em: 19 ago, 2021. [Online]. Disponível: https://canalte ch.com.br/empresa/yahoo/\\[0.5cm] [28] Canaltech, ``MSN", Acesso em: 19 ago, 2021. [Online]. Disponível: https:// canaltech.com. br/empresa/msn/\\[0.5cm] [29] Estadão, ``Ha 20 anos grupo estado entrava na internet", Acesso em: 19 ago, 2021. [Online]. Disponível: http://m.acervo.estadao.com.br/noticias/acervo,ha-20-anos--grupo-estado-entrava-n a-internet,10781,0.htm\\[0.5cm] [30] Wikipédia - A enciclopédia livre, ``R7", Acesso em: 19 ago, 2021. [Online]. Disponível: https://pt.wikipedia.org/wiki/R7\\[0.5cm] [31] UOL, ``História da Folha", Acesso em: 19 ago, 2021. [Online]. Disponível: https://www1.fol ha.uol.com.br/folha/circulo/historia\textunderscore folha.htm\\[0.5cm] [32] UOL, ``História da Folha", Acesso em: 19 ago, 2021. [Online]. Disponível: https://www1.fol ha.uol.com.br/institucional/historia\textunderscore da\textunderscore folha.shtml?fill=4\\[0.5cm] [33] Wikipédia - A enciclopédia livre, ``Terra (empresa)", Acesso em: 20 ago, 2021. [Online]. Disponível: https://pt.wikipedia.org/wiki/Terra\textunderscore (empresa)\\[0.5cm] [34] UOL, ``História - Sobre UOL", Acesso em: 20 ago, 2021. [Online]. Disponível: https://sobreu ol.noticias.uol.com.br/historia/\\[0.5cm] [35] C. A. O. Rodrigues, ``Acessibilidade digital da pessoa com deficiência" em JusBrasil, 2016.\\
\end{titlepage}
\end{document}