\documentclass[a4paper]{article}

\usepackage[brazil]{babel}
\usepackage[utf8]{inputenc}
\usepackage{amsmath,amsfonts,amssymb,latexsym,mathrsfs,amsthm,amstext,bezier,amscd}
\usepackage{graphicx}
\usepackage{indentfirst}

\begin{document}
\thispagestyle{empty}

\begin{titlepage}
	\vfill
	\begin{center}
		\parbox{6cm}{\includegraphics[scale=0.2]{logo.png}}\\
		\begingroup
		\fontsize{12pt}{0pt}\selectfont
		{\large \textbf{INSTITUTO FEDERAL DE EDUCAÇÃO, CIÊNCIA E TECNOLOGIA DO CEARÁ}}\\[0.2cm]
		\fontsize{12pt}{0pt}\selectfont
		{\large \textbf{IFCE CAMPUS MARACANAÚ}}\\[0.2cm]
		\fontsize{12pt}{0pt}\selectfont	
		{\large \textbf{BACHARELADO EM CIÊNCIA DA COMPUTAÇÃO}}\\[4cm]
		\fontsize{12pt}{0pt}\selectfont
		{\large \textbf{THIAGO MAGALHÃES FURTADO}}\\[3.5cm]
		\fontsize{12pt}{0pt}\selectfont
		{\large \textbf{Avaliação de Acessibilidade em Portais de Notícias}}\\[3.5cm]
		\fontsize{12pt}{0pt}\selectfont
		{\large \textbf{MARACANAÚ}}\\[0.2cm]
		\fontsize{12pt}{0pt}\selectfont
		{\large \textbf{2021}}
		\endgroup
	\end{center}
\end{titlepage}

\begin{titlepage}
	\vfill
	\begin{center}
		\fontsize{12pt}{0pt}\selectfont
		{\large \textbf{THIAGO MAGALHÃES FURTADO}} \\[2.5cm]
		\fontsize{12pt}{0pt}\selectfont
		{\large \textbf{Avaliação de Acessibilidade em Portais de Notícias}}\\[3cm]
		
		\hspace{.45\textwidth} %posiciona a minipage
		\begin{minipage}{.5\textwidth}
			\large Trabalho de Conclusão de Curso apresentado ao curso de Bacharelado em Ciência da Computação do Instituto Federal de Educação, Ciência e Tecnologia do Ceará (IFCE) - Campus Maracanaú, como requisito parcial para obtenção do Título de Bacharel em Ciência da Computação.\\[1cm]
			Prof. Dr. Otávio Alcântara de Lima Junior.
		\end{minipage}
		\vfill
		\vspace{2cm}		
		\large \textbf{Maracanaú - CE}
		
		\large \textbf{2021}
	\end{center}
\end{titlepage}

\begin{titlepage}
	\begin{center}
		\tableofcontents
	\end{center}
\end{titlepage}
\begin{titlepage}
\section{Introdução}
\fontsize{12pt}{0pt}\selectfont
Estamos vivendo em uma época em que a tecnologia tem avançado bastante, isso é notório se nós analisarmos as duas últimas décadas do século XX e o começo do século XXI, no qual o mundo entrou em uma nova era, conhecida como era da informação. Assim, os computadores, a internet, os acessos aos sites e o número de usuários em todo o mundo têm tido avanços significativos, possibilitando, para a nossa sociedade, uma prestação de serviço, facilitando, assim, a obtenção de informação rápida. Porém, muitos sites não possibilitam uma navegação agradável e tranquila, já que alguns usuários tem dificuldade para acessá-los, pelo simples motivo de possuírem alguma deficiência. Segundo a Patricia A.V., o Luís A. S. U. e o Sérgio L. M., a OMS disse que 15 por cento da população mundial tem alguma deficiência, como, baixa visão, pouca adição, diminuição da capacidade motora, dentre outros. Uma dessas categorias de sites, que são pouco acessíveis, são os portais jornalísticos, onde a procura de informação tem sido constante no contexto atual da nossa sociedade, porém, como outros sites, eles não possuem acessibilidade, não contemplando, assim, os PCDs, isto é, pessoas com deficiência.

Para analisarmos a acessibilidade de um site de notícia, são averiguados algumas formas de acessibilidade que são propostos por várias diretrizes existentes, como a WAI, a eMAG, porém a que é mais usada atualmente, e a que vamos usar em nossa pesquisa, é a diretriz WCAG(Web Content Accessibility Guidelines), que já está na sua segunda versão, isto é, a WCAG 2.0.

Segundo a Patricia A. V., a Tania A. e o Sérgio L. M., nela possui doze diretrizes organizadas em quatro princípios diferentes e algumas recomendações para cada princípio, que são indispensáveis em qualquer plataforma digital, inclusive os sites de notícia. Então iremos usá-la para classificar os portais jornalísticos como acessíveis ou não.

Um exemplo de pesquisa que podemos citar que utilizou a WCAG foi feita por Elisa M. M. L., Alejandro R. A., Emílio L., e Jorge P. M., onde avaliaram a acessibilidade dos vídeos existentes em uma plataforma online de curso chamada de MOOC. Foram analisados os vídeos através da verificação do uso de uma alternativa à informação visual, o uso de audiodescrição, incluindo a descrição das informações visuais na narração do vídeo. Além disso foi analisado o uso de transcrições e a utilização de uma fonte boa, com um tamanho de letra razoável e a existência de alto-contraste em toda a página web.

Portanto, este trabalho aborda a prática do uso da acessibilidade de dez sites de notícia para pessoas que possuem algum tipo de deficiência, seja ela visual, auditiva, física ou motora, com o objetivo de analisá-los e classificá-los, como acessíveis ou não, através da análise da diretriz WCAG sobre acessibilidade, averiguando a existência de autocontraste, aumentar e diminuir fonte, vídeos de libras, audiodescrição, disponibilidade para o uso do leitores de tela, existência de atalhos para as funcionalidades dos sites, dentre outros, com o objetivo de incentivar todas as pessoas que estão envolvidas na formulação dessas plataformas para deixá-las cada vez mais acessíveis.

Isso será organizado durante todo o artigo da seguinte a maneira, primeiro iremos falar sobre alguns artigos publicados anteriormente sobre o assunto, mostrando quais diretriz foram utilizadas nas suas pesquisas, em que área aconteceram os seus estudos e caracterizando a definição de acessibilidade digital por meio desse autores, mostrando o que cada um pensa sobre o assunto. Depois serão explicadas as formas de deficiência existentes que afetam a usabilidade dos sites, após isso iremos analisar a diretriz WCAG. Além disso, apresentaremos os pontos fortes e fracos dos sites de notícias relacionado com a acessibilidade web através da análise da diretriz com o propósito de fazer sugestões de melhorias. Por fim iremos tirar conclusões sobre o que foi levantado dos devidos sites.

\section{Revisão Bibliográfica}
Existem vários estudos e artigos que tratam sobre o tema da acessibilidade digital, artigos dos mais diversos possíveis, em várias regiões do mundo e sobre vários pontos de vista, assim iremos passar por alguns deles, mostrando qual ponto o mundo está em relação a acessibilidade web, quais pontos de melhoria são necessários nos sites em geral e quais conclusões cada autor conseguiu encontrar, focando em dois pontos principais, que serão as diretrizes utilizadas nas respectivas plataformas e sobre algumas propostas de avaliação em algumas áreas diferentes dos portais de notícia.

\subsection{Diretrizes}
O primeira referência que iremos citar são os autores o Daniel L. A., a Tatiane S. F., a Vânia M. F. C. e a Márcia Z. G., que definiram a acessibilidade digital, usando a Organização Mundial da Saúde(OMS), dizendo que, de acordo com a “Convenção sobre os direitos das pessoas com deficiência”, pessoas com deficiência são aquelas que têm incapacidades físicas, mentais, intelectuais ou sensoriais a longo prazo que, em interação com diversas barreiras, podem prejudicar sua participação plena e efetiva na sociedade em igualdade de condições com os outros. Segundo eles, no Brasil existe um decreto de número. 5.296 de 2004, que disponibilizou o Modelo de Acessibilidade em Governo Eletrônico (eMAG), que foi utilizado para analisado 107 sites de Instituições Federais de Educação(IFEs) do Brasil. Eles pontuaram alguns elementos importantíssimos para a acessibilidade tiradas da eMAG, como exemplo, atalhos de teclados, alto contraste, barra de acessibilidade, existência do mapa de sítio e páginas de descrição com os recursos de acessibilidade. Porém, concluíram que não foram implementados de maneira satisfatória os pontos de acessibilidade nos portais dos IFEs do país, e que as barreiras ao acesso à informação digital são o resultado da falta de atenção durante o processo de desenvolvimento da tecnologia, ao invés de limitações inerentes a essa tecnologia.

Outra diretriz existente, além da eMAG, é a World Wide Web (W3C), que foi criada em primeiro de outubro de 1994, no qual possui uma organização chamada World Wide Consortium, que é um consórcio com 450 membros, empresas, organizações independentes e ONGs com o objetivo de formular padrões para os desenvolvedores aos criarem os seus sites web, que foram formulados em uma diretriz chamada de WCAG. Segundo a Patricia A. V., a Tânia. A. e Sérgio L. M., citados anteriormente, falaram que a acessibilidade na web refere-se a recursos de design da web que permitem que as pessoas operem os sites, e que a WCAG 2.0, tem como objetivo eliminar erros de acessibilidade através dos web designers e dos desenvolvedores. Assim, ela servi de apoio para os desenvolvedores a boa prática da acessibilidade web, garantindo um acesso à internet satisfatório, atingindo o maior número de pessoas possível, independente da existência de uma comorbidade física. Por meio desse artigo, vemos que, para os seus autores, a criação da diretriz WCAG tem uma grande importância, pois, caso os programadores decidam seguir os seus princípios, tornarão as plataformas digitais existentes da web cada vez mais acessíveis para todo e qualquer ser humano que deseja usá-lo, independente se o usuário possui ou não qualquer um dos tipos de deficiência, seja ela física, motora ou mental. 

Ainda sobre a diretriz WCAG, o Hasan O. A. S. e o Mohammed A. A. descreveram, sucintamente, os seus quatro princípios. O primeiro princípio é o perceptível, no qual as informações e a interface devem ser apresentáveis aos usuários de uma forma que eles possam entender o conteúdo que está naquela plataforma, assim, é necessário que o site forneça alternativas para os conteúdos de textos, seja aumentar a letra, seja o uso de braile ou seja a existência de áudios e símbolos, com o intuito de utilizar outras linguagens. O segundo princípio é o operável, onde é definido métodos, para os usuários através da navegação, que possibilitem uma interação e a possibilidade de navegar pelo conteúdo confortavelmente, com o uso de interfaces apropriadas para os deficientes. O terceiro princípio é o compreensível, que diz que os usuários devem ser capazes de compreender as informações e o funcionamento da interface, com o objetivo do usuário interpretar corretamente o conteúdo da plataforma digital. Por fim, o quarto e último princípio é o robusto, que foca na grande variedade de usuários poder interpretar o conteúdo de maneira confiável, além de levar em consideração a compatibilidade com as tecnologias atuais e futuras, maximizando a harmonização das páginas da web com os seus usuários e com essas tais tecnologias.

Assim como dito anteriormente, um artigo produzido por Milton Campoverde-molina, Sergio Juan-Mora e Llorenç Valverde Garcia, no ano de 2020, diz que a acessibilidade na web visa tornar os sites mais acessíveis e utilizáveis pelo maior número de pessoas possível, independentemente de seus conhecimentos, habilidades ou características técnicas, e que a diretriz WCAG, criado pela empresa W3C, é essencial para desenvolvedores e organizações que desejam criar sites e ferramentas de alta qualidade, e não excluir as pessoas de usarem seus produtos e serviços. Então o artigo se propôs a falar sobre esses dois temas, que é explicar a diretriz e mostrar a definição da acessibilidade Web.

Então, podemos ver que as diretrizes são bem extensa e abrange quase todas as formas de acessibilidade existentes, agora, é preciso só que os programadores se propõem a seguir os quatro princípios básicos ditos anteriormente, com a meta de fazer tal atividade em seus trechos de códigos, fazendo com que todas as pessoas consigam manipular os seus sites, sem haver nenhum tipo de descriminação por partes dos programadores e dos produtores de conteúdos.

\subsection{Avaliação de Acessibilidade de Sites}
Agora iremos falar sobre alguns artigos que avaliaram os mais variado campos de sites existentes.  Vamos começar pelo autores que já foram citados anteriormente, que são o Daniel L. A., a Tatiane S. F., a Vânia M. F. C. e a Márcia Z. G., que avaliaram 107 sites educacionais no Brasil. Eles verificaram a existência de um grande descaso relacionado com a inclusão digital, pois apenas 14,02 por cento possuem atalhos no teclado, 22,43 por cento possuem a opção de alto contraste, 41,12 por cento possuem a barra de acessibilidade, 36,45 por cento possuem o mapa de sitio e só 14,02 por cento possuem páginas com descrições com os recursos de acessibilidade. Isso nos mostra que, segundo os pesquisadores, os resultados obtidos apontam para um elevado descaso com a inclusão digital em relação às pessoas com deficiência, de forma que o nível de adoção dos padrões de acessibilidade é extremamente baixo nos portais das Instituições Federais de Ensino, algo que é bastante ruim, pois nem mesmo os próprios sites governamentais possuem a acessibilidade, ainda que existam leis que exigem a acessibilidade nos sites.

\begin{center}
	\begin{tabular}{ccc}
		\hline \\[0.2cm]
		Características das Diretrizes & Primeira citação\\[0.2cm]
		\hline \\[0.2cm]
		Mapa de sítio & x \\[0.2cm]
		Alto-contraste &  x \\[0.2cm]
		Vídeo de libras &  \\[0.2cm]
		Áudios &  \\[0.2cm]
		Descrição de áudios e vídeos & x \\[0.2cm]
		Atalhos de teclados & x \\[0.2cm]
		Aumentar e diminuir fonte &  \\[0.2cm]
		Descrição de links &  \\[0.2cm]
		Disponibilidade para o leitor de tela &  \\[0.2cm]
		Vídeo de libras &  \\[0.2cm]
		Barra de acessibilidade & x \\[0.2cm]
		\hline
	\end{tabular}

\end{center}

Já um artigo produzido por Patricia A. V., Luis A. S. U. e Sérgio J. M., no ano de 2019, definiu o termo acessibilidade, dizendo que quando aplicado à web, diz respeito ao desenvolvimento de um design útil para facilitar o acesso a um número mais significativo de usuários. Uma página web acessível permitirá que os usuários com alguma deficiência permanente ou temporária recebam e entendam o conteúdo de um site, bem como possam navegar em tudo corretamente. Nesse artigo foi analisado um método heurístico existente para investigar o nível de acessibilidade de 40 sites, incluindo os de 30 universidades da América Latina e 10 sites entre os mais visitados, segundo o ranking Alexa. Foram analisados em relação aos usuários com baixa visão, o método utilizado foi proposto por Brajnik e WCAG 2.1, aplicando uma avaliação manual e se enquadra no grupo ''Testes de Triagem de Barreiras''. Esta técnica consiste em priorizar os impactos das barreiras de acordo com o contexto aplicado. O método permite a identificação da gravidade de cada barreira; este método heurístico busca identificar problemas de acessibilidade. A finalidade do artigo era analisar as barreiras existentes nas plataformas, com o objetivo de ajudar os programadores a ajeitarem esses impeditivos nessas plataformas, além de influenciar sites que ainda serão criados no futuro. Assim, o artigo concluiu que muitas dessas páginas web analisadas, violaram os princípios da diretriz WCAG, totalizando 241 barreiras que impossibilitam o manuseio simples e fácil do site.

\end{titlepage}
\end{document}